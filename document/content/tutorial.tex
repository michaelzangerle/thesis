\chapter{Tutorial}

\section{Images}

In der Abbildung \imgref{gull} k�nnen Sie eine M�we sehen.

\img{0.5}{document/graphics/gull.jpg}{Quelle: http://de.wikipedia.org/wiki/M�wen}{gull}

In der Abbildung \imgref{biene} k�nnen Sie eine Biene sehen:

\imgsrc{0.3}{document/graphics/biene.png}{Ich bin eine Bieene.}{biene}{frag google!}

\begin{code}[tex]{Inline code}{Inline code}
You can see a Gull in Abbildung \imgref{gull1}.

\img{0.5}{document/graphics/gull.jpg}{Quelle: http://de.wikipedia.org/wiki/Moewen}{gull1}

And a bee in Abbildung \imgref{biene1}:

\imgsrc{0.3}{document/graphics/biene.png}{Ich bin eine Bieene.}{biene1}{frag google!}
\end{code}

\section{Listings}

In Listing \coderef{example} k�nnen Sie ein Beispiel Code sehen.

\codefile[6][12][perl]{document/scripts/example.pl}{example caption}{example}

\begin{code}[tex]{Include a code file}{codefile}
In Listing \coderef{examplelabel} you can see a perl code sample (line 6 to 12).
\codefile[6][12][perl]{document/scripts/example.pl}{example caption}{example}
\end{code}

\begin{code}[c]{caption}{label}
#include <msp430.h> 

void main(void) {
	WDTCTL = WDTPW + WDTHOLD; // Stop watchdog timer

	P1DIR |= BIT0; // Set P1.0 to output direction
	P1OUT &= ~BIT0; // Set the red LED on

	TA0CCR0 = 12000; // Count limit (16 bit)
	TA0CCTL0 = 0x10;	 // Enable counter interrupts, bit 4=1
	TA0CTL = TASSEL_1 + MC_1; // Timer A 0 with ACLK @ 12KHz, count UP

	_BIS_SR(LPM0_bits + GIE); // LPM0 (low power mode) with interrupts enabled
}

#pragma vector=TIMER0_A0_VECTOR
   __interrupt void Timer0_A0 (void) { // Timer0 A0 interrupt service routine
	P1OUT ^= BIT0; // Toggle red LED
}
\end{code}


\begin{code}[tex]{Inline code}{Inline code}
In Listing \coderef{inline} you can see a c code sample.

\begin{code}[c]{example caption}{inline}
#include <msp430.h> 

void main(void) {
...
}
\ end{code}
\end{code}

\section{Bibtex}

Lorem ipsum dolor sit amet, consetetur sadipscing elitr, sed diam nonumy eirmod tempor invidunt ut labore et dolore magna aliquyam erat, sed diam voluptua. At vero eos et accusam et justo duo dolores et ea rebum. Stet clita kasd gubergren, no sea takimata sanctus est Lorem ipsum dolor sit amet. Lorem ipsum dolor sit amet, consetetur sadipscing elitr, sed diam nonumy eirmod tempor invidunt ut labore et dolore magna aliquyam erat, sed diam voluptua. At vero eos et accusam et justo duo dolores et ea rebum. Stet clita kasd gubergren, no sea takimata sanctus est Lorem ipsum dolor sit amet. \cite{bibtex.a}

Der Literaturgenerator macht es m�glich, schnell und einfach BibTeX-Eintr�ge zu erstellen.
Es reicht den Titel des Buches in das Suchfeld einzutragen.\\
\url{http://www.literatur-generator.de/}

Das ist ein Test! \cite{zheng2012}

Die meisten Bibliothekskataloge k�nnen mittlerweile BibTeX exportieren (ggf. nach \url{http://de.wikipedia.org/wiki/Citavi}). Hiermit k�nnen komplette Datens�tze in die BibTeX-Datenbank �bernommen werden.\\
Mit dem Webtool Lead2Amazon ist es m�glich, BibTeX-Eintr�ge f�r B�cher, die bei Amazon erh�ltlich sind, automatisch zu generieren.\\
In das Suchfeld muss dabei lediglich die ISBN eingetragen werden.\\
\url{http://lead.to/amazon/en/?op=bt}

Mit dem Literaturverwaltungsprogramm zotero lassen sich auch bibtex-Dateien exportieren.\\
\url{http://www.zotero.org}

\section{Algorithm}

This Algoorithm \algoref{algo1} is great to learn.

\begin{algo}{How to write algorithms}{algo1}
 \KwData{this text}
 \KwResult{how to write algorithm with \LaTeX2e }
 initialization\;
 \While{not at end of this document}{
  read current\;
  \eIf{understand}{
   go to next section\;
   current section becomes this one\;
   }{
   go back to the beginning of current section\;
  }
 }
\end{algo}

\begin{code}[Tex]{Algorithm code}{algorithm}
This Algoorithm \algoref{algo1} is great to learn.

\begin{algo}{How to write algorithms}{algo1}
 \KwData{this text}
 \KwResult{how to write algorithm with \LaTeX2e }
 initialization\;
 \While{not at end of this document}{
  read current\;
  \eIf{understand}{
   go to next section\;
   current section becomes this one\;
   }{
   go back to the beginning of current section\;
  }
 }
\end{algo}
\end{code}

\Blinddocument

\Blinddocument
