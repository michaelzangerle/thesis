\chapter{Segmentierung und Klassifizierung}
In diesem Abschnitt wird der Prozess der Klassifizierung von einzelnen Segmenten erklärt. Dies inkludiert den Segmentierungsvorgang sowie die Berechnung der Zusatzinformationen, welche für die Segmentierung benötigt werden. Weiters wird dargelegt, wie die Zusatzvariablen  ausgewählt worden sind. 

Unter dem Segmentierungsvorgang versteht man das Aufteilen einer GPS-Spur in Abschnitte, in welchen mit hoher Wahrscheinlichkeit nur ein Verkehrsmittel benutzt worden ist. Dabei wird versucht, die Segmente zu finden, in welchen man zu Fuß unterwegs war. Diese werden dann als Geh-Segmente gekennzeichnet. Die anderen Segmente werden als Nicht-Geh-Segmente klassifiziert. Da nur zwischen zwei temporären ``Verkehrsmitteln`` unterschieden wird, vereinfacht sich die  Verkehrsmittelbestimmung drastisch.

Nach der Segmentierung folgt die konkrete Bestimmung des jeweiligen Verkehrsmittels für die einzelnen Abschnitte, welche aus dem Segmentierungsvorgang hervorgegangen sind. In dieser Arbeit wurde die Bestimmung mit Hilfe von Entscheidungsbäumen realisiert. Dabei gibt es einen Entscheidungsbaum für die Klassifizierung mit und einen für die Klassifizierung ohne GIS-Daten. Für diese Entscheidungsbäume wurden verschiedene Entscheidungsvariablen ausgewählt. Sowohl die Auswahl als auch die Berechnung dieser Werte wird erklärt.

Schlussendlich werden die klassifizierten Abschnitte ein letztes Mal auf ihre Plausibilität in dieser Reihenfolge überprüft. Dabei stützt sich dieser Prozess auf die Aussage von Zheng \cite{zheng_understanding_2010}, nach welcher sich immer - wenn mitunter auch kurze - Geh-Segmente zwischen Verkehrswechseln befinden müssen und man nicht von einem in ein anderes Verkehrsmittel wechseln kann, ohne sich eine kurze Strecke zu Fuß bewegen zu müssen. Durch Miteinbeziehen des Kontexts lassen sich Abfolgen wie z.B. ``Auto->Bus->Auto->Bus->Auto`` verhindern.  

\section{Segmentierung eines Tracks}
\label{segmentierung}
Bei der Segmentierung geht es in erster Linie darum, den Umfang des Problems der Verkehrsmittelbestimmung zu verkleinern. Man versucht also, das angegebene GPX-Tracksegment in Geh-Segmente und Nicht-Geh-Segmente aufzuteilen. Dadurch muss man statt zwischen mehreren Verkehrsmitteln (Gehen, Bus, Zug, Auto, Fahrrad) nur mehr zwischen zwei unterscheiden. Die genauere Unterscheidung kann dann in einem weiteren Schritt folgen, in welchem jedoch schon klar ist, ob es sich um ein Segment handelt, in welchem eine Person zu Fuß unterwegs war oder nicht. 

Zheng sagt in diesem Zusammenhang, dass es zwischen jedem Wechsel eines Verkehrsmittels einen Abschnitt gibt, in welchem man zu Fuß unterwegs war und ein Stopp stattgefunden hat, auch wenn dieser Abschnitt sehr klein war. Ein Wechsel erfolgt nie direkt, wie z.B. von einem Zug in den Bus ohne anzuhalten und über einen Bahnsteig gehen zu müssen. Wenn also, ein Verkehrsmittelwechsel stattgefunden hat, dann gibt es neben Geh-Punkten auch Trackpunkte, in welchen eine Geschwindigkeit und eine Beschleunigung von nahezu null zu erwarten ist. Diese Aussage hat sich aus seinen umfangreichen Testdaten ableiten lassen. Daraus hat er folgenden Algorithmus abgeleitet:  \cite{zheng_understanding_2010}

\begin{pitemize}
\item Finde alle Geh-Punkte und Nicht-Geh-Punkte des betrachteten Abschnitts oder Tracks anhand von Grenzwerten für Geschwindigkeit und Beschleunigung 
\item Fasse die aufeinander folgenden Punkte vom selben Typ in Segmente zusammen.
\item Liegt die Distanz oder die Zeit eines solchen Segments unterhalb einer definierten Grenze, so vereine dieses Segment mit dem vorherigen.
\item Überschreitet ein Segment eine bestimmte Länge (200 Meter), ist es ein ``sicheres Segment``. Liegt die Distanz eines Segments jedoch unterhalb dieses Werts, so ist es ein ``unsicheres Segment``. Überschreitet die Anzahl der aufeinander folgenden ``unsicheren Segmente`` einen bestimmten Grenzwert, so werden die  aufeinander folgenden ``unsicheren Segmente`` vereint und als Nicht-Geh-Segment betrachtet.
\end{pitemize}

Biljecki ergänzt diesen Ansatz von Zheng mit seinen Erfahrungen, in welchen er feststellte, dass bei einem Wechsel des Verkehrsmittels oft auch das Signal verloren geht. Aus diesem Grund beendet Biljecki ein Segment, wenn die Verbindung verloren wurde und startet ein neues. Den Verbindungsverlust interpretiert Biljecki so, dass er für mindestens 30 Sekunden keinen weiteren Trackpoint findet. Weiters segmentiert Biljecki dann, wenn er für mehr als 12 Sekunden keine bzw. wenig Bewegung (< 2km/h) feststellen konnte. Diese beiden Änderungen am Algorithmus von Zheng begründet Biljecki damit, dass eine Übersegmentierung besser ist als eine Untersegmentierung. Aufeinander folgende Segmente vom Typ kann man immer noch in einem nächsten Schritt zusammenlegen. \cite{biljecki_transportation_2013}

Für den Prototyp wurde im Wesentlichen der durch Biljecki erweiterte Ansatz von Zheng gewählt. Vorallem das Einteilen der Segmente in ``sichere`` und ``unsichere`` Segmente hat sich bei der Bestimmung des Verkehrsmittels als essentiell herausgestellt. Liegt die Distanz  des Segments unterhalb einer gewissen Grenze, so tendiert die Klassifizierung dazu, falsche Entscheidungen zu treffen. Speziell im Kontext mit  GIS-Daten konnte man feststellen, dass diese ihre Wirkung erst ab einer bestimmten Distanz entfalten konnten (z.B. mehrere kurze Segmente mit jeweils einem Stopp bei einer Bushaltestelle im Gegensatz zu einem längeren Segment mit vielen Stopps bei Bushaltestellen).

 Das Resultat eines Segmentierungsvorgangs kann man in Abbildung \imgref{segmentierung} sehen. Jeder Kreis auf der eingezeichneten Route zeigt den Start bzw. das Ende eines Segments. Auf Basis dieser ersten Aufteilung kann nun die genauere Bestimmung des Verkehrsmittels erfolgen.

\img{0.65}{document/graphics/segmentierung.png}{Grundlegende Segmentierung}{segmentierung}

\section{Schlussfolgerungsvariablen}
\label{schlussfolgerungsvariablen}

Als Grundlage für die Schlussfolgerung durch die Entscheidungsbäume müssen zuerst Variablen festgelegt werden, die möglichst ausschlaggebend für die betrachteten Verkehrsmittel sind. Diese Schlussfolgerungs- oder auch Entscheidungsvariablen sind ein Teil einer Bedingung in einem Knoten eines Entscheidungsbaums. Eine Bedingung besteht dabei aus der Variable (z.B. Stopprate), einem Operator (<= oder >) sowie einem Wert. Wird ein Baum mit einem Segment traversiert, so muss an jedem Knoten eine Entscheidung gefällt werden, welche dann vorgibt, ob mit dem linken oder rechten Kind fortgefahren werden soll. Dies wird gemacht, bis ein Blatt im Baum erreicht wird. Ein Blatt enthält das Resultat für die  getroffenen Entscheidungen und gibt Auskunft darüber, zu wie viel Prozent Wahrscheinlichkeit ein Segment von einem bestimmten Transporttyp ist.

Diese Variablen können dabei diverse geschwindigkeitsabhängige Werte sein, wie durchschnittliche und maximale Geschwindigkeit oder auch Beschleunigungswerte und  Abstände zu bestimmten Infrastrukturen. Eine Übersicht über die Schlussfolgerungsvariablen in den betrachteten Publikationen ist in der Tabelle \ref{variablenuebersicht} abgebildet. Diese Werte sind nach der Anzahl der Vorkommnisse in den Publikationen gereiht und bilden eine Grundlage für die in dieser Arbeit verwendeten Variablen. 

Aus Gründen der Vollständigkeit muss noch erwähnt werden, dass aufgrund eines anderen Segmentierungsverfahrens nicht alle Variablen von Gonzales  \cite{gonzalez_automating_2010} aufgeführt sind. Außerdem handelt es sich bei der maximalen Geschwindigkeit nicht um das absolute Maximum, sondern um 95\% davon bzw. die dritthöchste Geschwindigkeit, die gemessen wurde. 

Weiters muss noch erwähnt werden, dass sowohl Zheng  \cite{zheng_understanding_2010} als auch Stenneth \cite{stenneth_transportation_2011} nicht alle Variablen schlussendlich verwendet haben. Sie haben allerdings evaluiert, welche Variablen in ihrem Projekt die größte Wirkung zeigen. Dies war bei Zheng \cite{zheng_understanding_2010} die Kombination der Stopprate mit der Geschwindigkeitsänderungsrate und der Richtungsänderungsrate. Fügte er weitere Variablen hinzu, konnte er eine Verschlechterung der Ergebnisse beobachten. Bei Stenneth \cite{stenneth_transportation_2011} hat die Evaluierung ergeben, dass die durchschnittliche Geschwindigkeit und Beschleunigung kombiniert mit verschiedenen GIS-Werten das beste Ergebnis liefert. 

Die betrachteten GIS-Werte waren bei Stenneth \cite{stenneth_transportation_2011} der durchschnittliche Abstand zu Gleisen, Bussen und dem Buskandidaten (jenen Bus, in dem sich die Person am ehesten befand). Biljecki verwendete den Abstand zu Gleisen, Bushaltestellen, Buslinien, U-Bahn und Straßenbahn als Indikatoren für die Schlussfolgerung. Schüssler \cite{nadine_schussler_improving_2011} inkludiert von den möglichen GIS-Werten nur den Abstand zu öffentlichen Verkehrsmitteln aller Art.

\subsection{Reihung der allgemeinen Variablen}
Die in dieser Arbeit verwendeten Schlussfolgerungsvariablen basieren auf der Reihung, welche in Tabelle \ref{variablenuebersicht} ersichtlich ist. Die allgemeine Geschwindigkeit an fünfter Stelle wurde deshalb übersprungen, weil sie bereits zwei mal vertreten ist und bereits Zheng gesagt hat, dass die Geschwindigkeit allein kein aussagekräftiger Indikator für ein Verkehrsmittel ist \cite{zheng_understanding_2010}. Weiters hat Zheng für die Auswahl der Stopprate eine interessante Erklärung, welche im Abschnitt ``\ref{stopprate} \nameref{stopprate}`` genauer erklärt wird. Deshalb wurde die Stopprate auch als fünfte Variable ausgewählt. 

\begin{enumerate}
\item durchschnittliche Geschwindigkeit
\item maximale Geschwindigkeit
\item durchschnittliche Beschleunigung
\item maximale Beschleunigung
\item Stopprate
\end{enumerate}


\subsection{Reihung der GIS-Variablen}
Bei der Auswahl der GIS-Variablen fallen jene mit U-Bahn und Straßenbahn weg, da es diese im Raum Vorarlberg nicht gibt. Jene Variablen, die spezifische GPS-Daten von einzelnen Verkehrsmitteln benötigen, konnten aufgrund von fehlenden Schnittstellen zu den Daten der öffentlichen Verkehrsbetriebe nicht verwendet werden. Somit wurden folgende Variablen für die GIS-gestützte Analyse verwendet:

\begin{pitemize}
\item durchschnittliche Nähe zu Busstationen und Bahnhöfen
\item durchschnittliche Nähe zu Gleisen	
\item durchschnittliche Nähe zu Autobahnen
\end{pitemize}
\clearpage

\begin{landscape}
\begin{table}[h]
\centering
\begin{tabular}{|l|c|c|c|c|c|c|c|}
\hline
 & \multicolumn{1}{l|}{Zheng\textsuperscript{1}} & \multicolumn{1}{l|}{Stenneth\textsuperscript{2}} & \multicolumn{1}{l|}{Reddy\textsuperscript{3}} & \multicolumn{1}{l|}{Biljecki\textsuperscript{4}} & \multicolumn{1}{l|}{Gonzales\textsuperscript{5}} & \multicolumn{1}{l|}{Schüssler\textsuperscript{6}} & \multicolumn{1}{l|}{\textbf{Gesamt}} \\ \hline
durchschn. Geschwindigkeit & x & x &  & x & x & x & \textbf{5} \\ \hline
max. Geschwindigkeit * & x &  &  & x & x &  & \textbf{3} \\ \hline
durchschn. Beschleunigung &  & x & x &  & x &  & \textbf{3} \\ \hline
verwendet GIS Daten &  & x &  & x &  & x & \textbf{3} \\ \hline
Geschwindigkeit &  &  & x &  &  & x & \textbf{2} \\ \hline
max. Beschleunigung & x &  &  &  & x &  & \textbf{2} \\ \hline
Richtungswechselrate & x & x &  &  &  &  & \textbf{2} \\ \hline
Stopprate & x &  &  &  &  &  & \textbf{1} \\ \hline
Distanz des Segments & x &  &  &  &  &  & \textbf{1} \\ \hline
Distanz des Tracks &  &  &  &  & x &  & \textbf{1} \\ \hline
Geschwindigkeitswechselrate & x &  &  &  &  &  & \textbf{1} \\ \hline
durchschn. Varianz d. Beschl. &  &  &  &  &  & x & \textbf{1} \\ \hline
durchschn. beweg. Geschw. &  &  &  & x &  &  & \textbf{1} \\ \hline
durchschn. Genauigkeit &  & x &  &  &  &  & \textbf{1} \\ \hline
erwartete Geschwindigkeit & x &  &  &  &  &  & \textbf{1} \\ \hline
Varianz d. Geschwindigkeit & x &  &  &  &  &  & \textbf{1} \\ \hline
\end{tabular}
\caption{Entscheidungsvariablenübersicht}
\textsuperscript{1} \cite{zheng_understanding_2010}, \textsuperscript{2} \cite{stenneth_transportation_2011}, \textsuperscript{3} \cite{reddy_using_2010}, \textsuperscript{4} \cite{biljecki_transportation_2013}, \textsuperscript{5} \cite{gonzalez_automating_2010}, \textsuperscript{6} \cite{nadine_schussler_improving_2011}
\label{variablenuebersicht}
\end{table}
\end{landscape}

\section{Berechnung der allgemeinen Entscheidungsvariablen}
Das Hinzufügen und Berechnen der Entscheidungsvariablen ohne GIS-Daten (z.B. Geschwindigkeit aus Weg und Zeit), wird vom Tracksegment-Filter (siehe Abschnitt ``\ref{tracksegmentFilter} \nameref{tracksegmentFilter}``) realisiert. Dabei ist die Berechnung der grundlegenden Geschwindigkeit zwischen zwei GPS-Punkten essentiell, da alle weiteren Entscheidungsvariablen darauf aufbauen. 

\subsection{Geschwindigkeit}
Berechnet wurde die Geschwindigkeit (v) als Durchschnittsgeschwindigkeit, die als Verhältnis vom zurückgelegten Weg (s) zur Zeit (t) ausgedrückt wird, wie es z.B. im Buch ``Physik`` von Douglas Giancoli definiert wird \cite[S.~27]{douglas_giancoli_physik_2010} und in \ref{geschwindigkeitsformel} ersichtlich ist. 
\begin{equation}
v = \frac{s}{t}
\label{geschwindigkeitsformel}
\end{equation}

Die Zeit ist dabei die Differenz zwischen den Zeitstempeln von zwei GPS-Trackpunkten und der Weg die Differenz zwischen den zwei Koordinaten. Zur Berechnung des Weges (der Luftlinienentfernung) zwischen zwei Koordinaten gibt es laut \cite{movable_type_ltd_calculate_2015} mehrere Möglichkeiten. Welche man verwendet, hängt von den Längen der betrachteten Strecken, der gewünschten Genauigkeit sowie der benötigten Performanz ab (schnell, aber ungenau vs. langsam, aber genau). Da es sich bei den hier betrachteten Distanzen immer um sehr kleine Distanzen im Verhältnis zur Erde selbst handelt, aber diese aufgrund ihrer weiteren Verwendung zur Berechnung der Geschwindigkeit möglichst genau sein soll, wird in diesem Prototyp die Haversine-Formel (siehe die Gleichungen \ref{haversineformel}) verwendet. Dabei sind $\varphi$ die Breitengrade, $\lambda$ die Längengrade, R der Erdradius und $\triangle\varphi$ bzw. $\triangle \lambda$ die Differenz der Breiten- bzw. Längengrade.\cite{movable_type_ltd_calculate_2015}

\begin{equation}
\label{haversineformel}
\begin{aligned}
a &= sin^{2}(\triangle \varphi/2) + cos \varphi_1  * cos \varphi_2 * sin^{2}(\triangle \lambda/2) \\
c &= 2 * atan2( \sqrt{a}, \sqrt{(1-a)}) \\
d &= R * c
\end{aligned}
\end{equation}

\paragraph{Durchschnittliche Geschwindigkeit} Die durchschnittliche Geschwindigkeit wird aus allen Geschwindigkeitswerten eines Segments berechnet.

\paragraph{Maximale Geschwindigkeit} Die maximale Geschwindigkeit wird aus allen Geschwindigkeitswerten des Segments ermittelt.

\subsection{Beschleunigung}
Die Beschleunigung ist als Geschwindigkeitsunterschied zwischen zwei Punkten pro Zeiteinheit definiert \cite[S.~51]{douglas_giancoli_physik_2010}. Somit wurde diese auf Basis der ermittelten Geschwindigkeit berechnet und dadurch wurden folgende zwei Entscheidungsvariablen bestimmt:

\begin{pitemize}
\item \textbf{Durchschnittliche Beschleunigung:} die durchschnittliche Beschleunigung für das betrachtete Segment. 
\item \textbf{Maximale Beschleunigung:} die maximal gemessene Beschleunigung für das betrachtete Segment.
\end{pitemize}

\subsection{Stopprate}
\label{stopprate}
Wie Zheng in der Publikation  \cite{zheng_understanding_2010} beschrieben hat, ist die Geschwindigkeit als Basis für Entscheidungsvariablen nur bedingt geeignet, da diese sehr von der aktuellen Verkehrssituation abhängig ist. Deshalb setzt Zheng auf die Kombination von Richtungswechsel, Geschwindigkeitswechsel und der Stopprate. Dieser Aussage widerspricht wiederum indirekt die Auswahl der Entscheidungsvariablen von anderen Autoren wie Biljecki  \cite{biljecki_transportation_2013}, Gonzales \cite{gonzalez_automating_2010}, Schüssler \cite{nadine_schussler_improving_2011} und Reddy \cite{reddy_using_2010} (siehe Tabelle \ref{variablenuebersicht}). 

Die Stopprate ist für Zheng insofern sehr aussagekräftig, da er an spezifischen Verkehrsmitteln auch eine spezifische Stopprate bzw. ein Stoppverhalten festmachen kann. Dies ist beispielhaft in Abbildung \imgref{stop_rate_zheng} ersichtlich. Dabei stoppt eine Person im Auto (a) wesentlich weniger oft wie zum Beispiel ein Bus (b), da dieser nicht nur bei Kreuzungen sondern auch bei Bushaltestellen anhalten muss. Noch öfter stoppt laut Zheng nur ein Fußgänger (c) \cite{zheng_understanding_2010}. Im Falle des Prototyps wird die Erkennung von Stopps bereits beim Segmentieren benötigt. Da die anderen Entscheidungsvariablen auf der Geschwindigkeit basieren, wurde die Stopprate als weitere Entscheidungsvariable, die nicht abhängig von der Geschwindigkeit ist, ausgewählt. Als Stopp gilt dabei eine bestimmte Anzahl von Trackpunkten (z.B. über eine Zeit von 5 Sekunden), bei welchen die Geschwindigkeit unterhalb eines Grenzwerts (z.B. unter 2km/h) liegt. Diese Werte können wiederum in der Konfiguration eingestellt werden.

\img{0.55}{document/graphics/stop_rate_zheng.png}{Stopprate laut Zheng (Quelle: \cite{zheng_understanding_2010})}{stop_rate_zheng}
\clearpage

\section{Berechnung der GIS-Entscheidungsvariablen}
\label{gisdaten}
Das Hinzufügen und Berechnen der Entscheidungsvariablen mit GIS-Daten wird vom GISTracksegment-Filter (siehe Abschnitt ``\ref{gisTracksegmentFilter} \nameref{gisTracksegmentFilter}``) realisiert. Dabei erweitert der GISTracksegment-Filter den Tracksegment-Filter und ergänzt diesen mit den in Abschitt ``\ref{abstand-bhst} \nameref{abstand-bhst}`` und in Abschnitt``\ref{abstand-gleis-autobahn} \nameref{abstand-gleis-autobahn}`` beschriebenen GIS-Werten. Bei allen GIS-Werten wird eine Bounding-Box berechnet und auf die zuvor in die Datenbank importierten GIS-Daten zugegriffen. Der Radius der Boundingbox lässt sich in der Konfiguration festlegen.

\subsection{Abstand zu Bushaltestellen}
\label{abstand-bhst}
Da der Beginn und das Ende eines Segments immer ein Stopp ist, wird hierbei auch überprüft, ob sich eine Bushaltestelle in der Nähe befindet. Wird eine Bushaltestelle gefunden, so wird diese in den GIS-Wert für die Bestimmung des Verkehrsmittels miteinbezogen. Ist sowohl der Beginn als auch das Ende eines Segments in der Nähe einer Bushaltestelle, so wird dies höher gewichtet. 
Innerhalb eines Segments können sich weitere kürzere Stopps befinden. Diese werden zusätzlich verwendet, um festzustellen, ob Bushaltestellen in der Nähe sind und es sich im Endeffekt um einen Bus handelt der von Station zu Station fährt. Im Gegensatz zu den Haltestellen am Beginn und am Ende eines Segments werden diese allerdings nicht höher gewichtet, da sich Bushaltestellen oft in der Nähe von Ampeln oder Kreuzungen befinden. Dies stellte auch Biljecki \cite{biljecki_transportation_2013} schon fest.

\subsection{Abstand zu Gleisen und zur Autobahn}
\label{abstand-gleis-autobahn}
Ähnlich wie bei den Bushaltestellen werden auch die GIS-Werte für die Nähe zu Gleisen und der Autobahn am Beginn und am Ende eines jeden Segments berechnet. Allerdings wird für die Überprüfung innerhalb eines Segments ein konfigurierbarer Zeitwert verwendet. Im Prototyp wird alle 20 Sekunden überprüft ob sich ein Punkt in der Nähe eines Gleises oder der Autobahn befindet.

\section{Klassifizierung der Verkehrsmittel}
Die Klassifizierung der Segmente bzw. das Bestimmen des Verkehrsmittels für ein Segment basiert bei beiden Analysearten (mit und ohne GIS-Daten) auf Entscheidungsbäumen. Wie bereits in Abbildung \imgref{decisiontree} und in Abbildung \imgref{decisiontree-gis} ersichtlich und im Abschnitt ``\ref{schlussfolgerungsvariablen} \nameref{schlussfolgerungsvariablen}`` genau beschrieben, verwendet der Entscheidungsbaum ohne GIS-Daten ausschließlich Geschwindigkeits- und Beschleunigungswerte sowie die Stopprate als Indikatoren. Der Entscheidungsbaum mit GIS-Daten fügt diesen aber noch einen Wert für die Nähe zu verschiedenen Infrastrukturen hinzu.


\subsection{Erstellen der Entscheidungsbäume}
\label{entscheidungsbaumGenerierungPHP}
Wie bereits im Abschnitt ``\ref{entscheidungsbaumAlsSchlussfolgerungsmodell} \nameref{entscheidungsbaumAlsSchlussfolgerungsmodell}`` beschrieben, konnte in der freien Version von RapidMiner der generierte Entscheidungsbaum als Bild oder Text exportiert werden. Um jedoch beim Erstellen des Entscheidungsbaums möglichst flexibel zu sein (z.B. Erweitern der Trainingsdaten), wurde ein Parser für die textuelle Darstellung des Entscheidungsbaums implementiert. 

Ein Ausschnitt des Entscheidungsbaums als Text ist in Listing \coderef{entscheidungsbaumText} zu sehen. Dabei stellt jede Zeile mindestens einen Knoten im Baum dar. Eine Zeile kann mit Indikatoren für die Tiefe des Knotens im Baum beginnen. Dabei steht jedes ``|`` für eine Ebene tiefer im Baum. Jene zwei Zeilen am Beginn ohne Tiefenangabe bilden somit den Wurzelknoten, da sie über keine Tiefenangabe verfügen. Außerdem gibt es jeden Knoten, der kein Blatt ist, zwei mal in dieser Darstellung. Diese beiden Knoten unterscheiden sich dabei nur durch ihren Vergleichsoperator, denn dieser ist das genaue Gegenteil des anderen. Dadurch können beide Fälle einer Bedingung abgebildet werden.

Jede Zeile besteht aus dem Namen der Entscheidungsvariable, einem Vergleichsoperator und einem Wert. Hat ein Knoten nur mehr einen Kind-Knoten mit dem Resultat, so folgt hinter dem Wert das Resultat. Das Resultat besteht dabei aus dem bestimmten Verkehrsmittel und einer Übersicht über die Vorkommnisse aller Verkehrsmittel für die getroffenen Entscheidungen aus den Trainingsdaten.

Alle Kind-Knoten sind jeweils dem letzten Knoten auf dem vorherigen Level zuzuordnen.

\begin{code}[]{Entscheidungsbaum in Textform}{entscheidungsbaumText}
meanvelocity > 20.830: train {bike=0, walk=0, car=0, bus=0, train=5}
meanvelocity <= 20.830
|   meanvelocity > 2.041
|   |   meanvelocity > 7.837
|   |   |   meanvelocity > 8.772: car {bike=0, walk=0, car=46, bus=1, train=2}
|   |   |   meanvelocity <= 8.772
|   |   |   |   meanacceleration > 2.728: bus {bike=0, walk=0, car=0, bus=2, train=0}
|   |   |   |   meanacceleration <= 2.728: car {bike=0, walk=0, car=3, bus=0, train=0}
...
\end{code}

\textbf{Parsen und Cachen der Entscheidungsbäume} \label{parsenCachenEntscheidungsbaum}\\
Die von RapidMiner gelieferten Entscheidungsbäume wurden im Prototyp als Teil der Konfiguration hinterlegt. Damit die Entscheidungsbäume nicht für jede Anfrage geparst und erstellt werden müssen, bietet das Framework Symfony die Möglichkeit, die aus Konfigurationsdateien entstandenen Resultate zu cachen. Das bedeutet in diesem Fall, dass die Entscheidungsbäume eingelesen sowie geparst werden und danach mit Hilfe des Resultats eine PHP-Datei erstellt wird. Darin wird die Initialisierung des Entscheidungsbaums als Code abgelegt (siehe Listing \coderef{entscheidungsbaumPHP}). Dadurch muss der Entscheidungsbaum nicht jedes Mal neu geparst, sondern nur ein Objekt mit genau diesem Baum instanziiert werden. 

\begin{code}[PHP]{Ausschnitt des generierten Entscheidungsbaums als PHP-Klasse}{entscheidungsbaumPHP}
...
class BasicDecisionTree implements DecisionTreeInterface
{
    protected $tree;

    function __construct()
    {
        $node0 = new Node();
        $node0->setDecision(new Decision('meanvelocity', '>', 20.83));
        $node1 = new Node();
        $node1->setResult(new Result(0,0,0,0,5));
        ...
        $node0->setRight($node2);
        $node1->setParent($node1);
        $node2->setParent($node0);
        $node2->setLeft($node3);
        ...
\end{code}

Damit dieser Vorgang nur gemacht wird, wenn sich die Datei mit dem Entscheidungsbaum in Textform verändert, merkt Symfony sich das Änderungsdatum und entscheidet basierend darauf, ob die PHP-Datei neu generiert werden muss.

\subsection{Verwendung der Entscheidungsbäume}
\label{entscheidungsbaum_verwendung}
Die jeweiligen Entscheidungsbäume werden basierend auf der Analysemethode im Trav- el-Mode-Filter verwendet. Dabei wird für jedes Segment der Entscheidungsbaum traversiert und das Resultat beim jeweiligen Segment hinterlegt. Schlussendlich hat jedes Segment einen Verkehrsmitteltyp zugeordnet bekommen und wird zur Nachbearbeitung weitergegeben.

\subsection{Nachbearbeitung}
\label{nachbearbeitung}
Die Nachbearbeitung der Segmente mit dem bestimmten Verkehrsmittel wurde sowohl von Zheng \cite{zheng_understanding_2010} als auch Biljecki \cite{biljecki_transportation_2013} vorgeschlagen bzw. beschrieben. Dabei geht es darum, die klassifizierten Segmente auch im Kontext zu sehen. So kann laut Zheng \cite{zheng_understanding_2010} zum Beispiel nicht ein ``Auto``-Segment auf ein ``Bus``-Segment folgen, ohne dass sich zwischen ihnen ein Segment ``zu Fuß`` befindet (auch wenn dieses sehr kurz ist). Durch diese Aussage können sich Verkehrsmittelwechsel, die zum Beispiel aufgrund der Testdaten entstanden, aber nicht plausibel sind, beseitigen lassen. Dies bedeutet, dass das vorherige korrekt identifizierte Segment (im Sinne des Kontexts) als Ausgang für das nächste Segment verwendet wird und dabei auch der Verkehrsmitteltyp des vorherigen Segments übernommen wird. 

Ein Beispiel dafür ist in Abbildung \imgref{nopostprocessing-vs-postprocessing} zu sehen. Links sieht man dabei den analysierten Track ohne Nachbearbeitung und rechts mit Nachbearbeitung. Die Abfolge der Verkehrsmittel links ist dabei Bus->Auto->Fahrrad->Auto->Fahrrad->Bus->Auto, was schlicht nicht möglich ist aber aufgrund der Trainingsdaten und des fehlenden Kontexts entstanden ist. Am Ende der Nachbearbeitung ist das korrekte Resultat eine durchgängige Busfahrt. 

\img{1}{document/graphics/postprocessing.png}{Kein Nachbearbeiten vs. Nachbearbeiten}{nopostprocessing-vs-postprocessing}

Eine weitere Aufgabe der Nachbearbeitung ist das herausfiltern von einzelnen Fahrradsegmenten. Diese sind aufgrund der vielen unterschiedlichen Fahrrad-Trainingsdaten entstanden. In einigen Fällen konnte es dadurch vorkommen, dass Segmente als Fahrrad-Segment identifiziert wurden, diese aber für sich alleine standen wie z.B. zu Fuß->Fahrrad->Bus->Bus->Bus. Aus diesem Grund wurde das Nachbearbeiten dahingehend erweitert, dass einzelne Fahrradsegmente herausgefiltert werden bzw. der Typ des Segments mit dem des vorherigen oder nächsten Segments abgestimmt wird.
\clearpage