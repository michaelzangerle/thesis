\chapter{Einleitung}
Der Mensch produziert täglich eine extrem große Menge an Daten. Allein auf Youtube werden pro Minute 300 Stunden Video-Material veröffentlicht und täglich hunderte Millionen Stunden Videos konsumiert.  \cite{youtube_statistics_2015} Hochgerechnet auf das gesamte Internet und die gesamte Bevölkerung ergibt dies eine unvorstellbar große Menge an Daten, die bewusst oder auch unbewusst generiert werden. 

Sehr viel an Daten wird auch durch diverse Fitnessgadgets, Smartwatches, Smartphones sowie Navis und ähnliche Geräte generiert. Abseits von Fitnesswerten sind viele dieser Geräte im Regelfall in der Lage GPS-Spuren aufzuzeichnen. Dies bedeutet, dass man genau nachvollziehen kann, wann man wo unterwegs war. Mit ein wenig Rechenarbeit können auch die Geschwindigkeit und viele andere Werte berechnet werden, sofern dies die Geräte nicht schon selbst machen. 

Genau auf diesen GPS-Daten basiert diese Arbeit. Dabei ist es nicht wichtig, von welcher Person diese Daten stammen, sondern dass sich mit Hilfe dieser aufgezeichneten Daten feststellen lässt, wann ein Individuum sich auf welcher Strecke mit welchem Verkehrsmittel fortbewegt hat.

Analysiert man viele dieser Daten, so lassen sich viele Erkenntnisse daraus gewinnen. Unter anderem lassen sich zum Beispiel viel frequentierte Strecken in der Infrastruktur finden und mögliche Engstellen erkennen. In Folge lassen sich damit auch mögliche Verbesserungen, Einsparungsmöglichkeiten oder auch verstecktes Potential für zukünftige Projekte entdecken.

\section{Ziele dieser Arbeit}
Das Hauptziel dieser Arbeit ist es, einen Prototypen zu erstellen, welcher anhand von aufgezeichneten GPS-Spuren und in Kombination mit verschiedenen Methodiken das benutzte Verkehrsmittel mit einer möglichst hohen Wahrscheinlichkeit bestimmt. Diese aufgezeichneten GPS-Spuren enthalten dabei keinerlei Informationen über die jeweilige Person. Deshalb erfolgt die Auswertung ausschließlich über die individuelle GPS-Spur sowie über öffentlich zugängliche Daten, wie zum Beispiel Busstationen und Gleise. Jede dieser Aufzeichnung kann dabei mehrere Verkehrsmittel beinhalten und von unterschiedlicher Länge und Dauer sein.

Die in dieser Arbeit berücksichtigten Verkehrsmittel sind in folgender Auflistung ersichtlich:

\begin{pitemize}
\item Fußgänger
\item Fahrrad
\item Bus
\item Auto (stellvertretend für Motorrad, Taxi, LKW, PWK etc.)
\item Zug
\end{pitemize}

Besonders interessant ist hierbei die Unterscheidung zwischen Bus und Auto in verkehrsabhängigen Situationen bzw. in der Stadt, da diese Transporttypen in diesen Situationen sehr ähnliche Verhaltensweisen und Werte zeigen. 

Es soll weiters für jede Person, die ein Gerät besitzt, das in der Lage ist, eine GPS-Spur im GPX-Format aufzuzeichnen, möglich sein, diese Spur analysieren zu lassen. Dies bedeutet, dass keine speziellen Geräte oder andere Sensoren benötigt werden und dass sich dieser Prototyp mit möglichst geringen Anpassungen (Trainingsdaten, GIS-Daten und Konfigurationsparameter) auch in anderen Regionen anwenden lässt.

Im Zuge dieser Arbeit wird auch untersucht, welcher Grad an Genauigkeit sowohl mit als auch ohne geografische Zusatzinformationen im Raum Vorarlberg erreicht werden kann. Dabei soll es nach der Analyse die Möglichkeit geben, die automatisch bestimmten Verkehrsmittel manuell zu korrigieren, sollten diese nicht mit der Realität übereinstimmen. Weiters sollen diese manuellen Änderungen in die Auswertung mit einfließen. 

Schlussendlich soll mit Hilfe der Auswertungen auch eine Aussage über die erzielte Genauigkeit mit und ohne die verwendeten Zusatzinformationen gemacht werden und untersucht werden, welche Zusatzwerte die Erkennungsrate wie beeinflussen. Außerdem soll erklärt werden, ob weitere Informationen (wie z.B. GPS-Daten der öffentlichen Verkehrsmittel) für eine noch genauere Bestimmung benötigt werden und welche Informationen dies sein könnten. 

\subsubsection{Nichtziele}

In dieser Arbeit werden die diversen Schlussfolgerungsmodelle (neuronales Netz, Bayessches Netz, Random Forest, Support Vector Machine, ...) nicht betrachtet oder verglichen, sondern es wird auf ein Modell gesetzt, das bereits bei anderen Arbeiten wie z.B. \cite{stenneth_transportation_2011}, \cite{reddy_using_2010}, \cite{sebastian_nagel_moglichkeitsstudie_2011} und  \cite{zheng_learning_2008} vielversprechende Ergebnisse erzielt hat. Dieses Modell ist der Entscheidungsbaum. Außerdem werden die Daten zur Analyse bzw. Auswertung nicht in Echtzeit betrachtet, sondern in Form einer GPS-Spur an den Prototypen übergeben.

Die Frage, an welcher Position man das Gerät zur Aufzeichnung am besten trägt, um möglichst genaue GPS-Daten zu erhalten, wird nicht weiter verfolgt, da die Daten möglichst realistisch sein sollen. Auch bleibt die Frage nach dem Energieverbrauch der App bzw. wie eine möglichst energieschonende App und die dazugehörige Kommunikation aufgebaut sein könnten, unberücksichtigt. Eine umfassende Behandlung der Themen Sicherheit und Privatsphäre ist im Rahmen der vorliegenden Arbeit nicht möglich und daher bleiben auch diese Themen unberührt. An dieser Stelle soll allerdings erwähnt werden, dass keine Informationen über den Benutzer/die Benutzerin in den GPS-Spuren benötigt oder vom Prototypen generiert oder gespeichert werden.

\section{Motivation und Nutzen}
Schon früher wurde versucht, Aufzeichnungen über die Verkehrswege von verschiedenen Menschen zu sammeln. Aber die Protokolle in Papierform sowie die Telefonbefragungen waren zu aufwändig und die teilnehmenden Personen nicht zuverlässig genug. Darum ist es von entscheidendem Vorteil, eine App oder ein Gerät zur Verfügung zu haben, welches die Vorgänge des Aufzeichnens möglichst genau  übernimmt. \cite{zheng_understanding_2010}

Wird eine Auswertung mit einer für das Zielgebiet aussagekräftigen Anzahl an Personen durchgeführt, so kann das Resultat  für verschiedenste Zwecke verwendet werden. Auch ohne spezielle Analyse kann rein durch die Betrachtung der gesammelten GPS-Spuren festgestellt werden, welche Routen besonders häufig benutzt werden.

Nimmt man nun verschiedene Werte aus der Auswertung hinzu, kann auch festgestellt werden, wo sich zum Beispiel verkehrstechnische Engstellen befinden und welche Routen sehr populär sind. Oder es können Aussagen über die allgemeine Verkehrssituation gemacht werden. Durch die gesammelten Daten lassen sich auch Simulationen für anstehende Bauvorhaben machen und es können Vorhersagen für bestimmte Situationen gemacht werden. Diese Aspekte können unter anderem für das Verkehrsministerium,  den öffentlichen Personennahverkehr oder auch für die Stadtplanung sehr interessant sein (Optimierung von Auslastung, Einsparungspotentiale, ...).

Eine ganze Reihe von Apps lässt sich mit den Auswertungen erstellen. Unter anderem könnten diese Apps die Auswertungen in soziale Medien integrieren oder für Fitnessanalysen verwendet werden. Einen einfachen Rückblick über die eigene Fortbewegung kann man damit genauso ermöglichen wie für Umweltbewusste errechnen, wie viel CO\textsubscript{2} sie produziert oder gespart haben. Ein Reisetagebuch könnte daraus genauso Nutzen ziehen wie eine App (sollten die Daten in Echtzeit ausgewertet werden), die beim Autofahren Auskunft über die aktuell billigste Tankstelle in näherer Umgebung gibt oder eine App, die einfach nur Vorschläge für alternative, schnellere Routen zu einem bekannten Ziel anbietet.

Zusammenfassend kann man sagen, dass die Verwendungsmöglichkeiten für solche Daten umfangreich sind und sich am besten unter dem Begriff kontextorientierte, geographische Applikationen  zusammenfassen lassen. Nicht zuletzt öffnen sich mit solchen Daten aber auch umfangreiche Möglichkeiten für die Werbebranche.

\section{Weiterer Aufbau der Arbeit}
Der Hauptteil der vorliegenden Arbeit gliedert sich in vier große Abschnitte:

Im ersten Abschnitt wird auf die Akquirierung der GPS-Daten eingegangen. Dies umfasst sowohl die gesammelten GPS-Aufzeichnungen und deren Struktur sowie die verwendeten GIS-Daten. Dabei geht es einerseits um deren Herkunft, andererseits auch darum, wie diese extrahiert wurden und wie diese Daten aufgebaut sind. Außerdem werden auch andere Daten, wie zum Beispiel GPS-Daten von Bussen des ÖPNV in Betracht gezogen.

Der zweite Abschnitt behandelt den entwickelten Prototypen, der die übergebenen GPS-Spuren analysiert. Dabei werden dessen Funktionalitäten erklärt und es wird der grundlegende Ablauf für den Benutzer/die Benutzerin dargelegt. Weiters wird auch auf die Architektur des Prototyps sowie auf dessen Konfigurationsmöglichkeiten eingegangen.

Der dritte Abschnitt befasst sich sowohl mit dem Einbinden der GIS-Daten als auch dem Aufbereiten der GPS-Daten sowie dem Bestimmen der Verkehrsmittel. Mit dem Aufbereiten der Daten ist gemeint, dass zu den GPS-Punkten zusätzliche Werte für die spätere Analyse berechnet werden. Beispiele für diese Werte sind sowohl die Geschwindigkeit, Beschleunigung und Distanz als auch der Abstand zur nächsten Bushaltestelle. 

Außerdem wird im dritten Abschnitt auch der Prozess des Aufteilens von GPS-Spuren in Teile (Segmente), in denen nur ein Verkehrsmittel verwendet wird, beschrieben. Diesem Schritt vorangegangen ist das in Anhang 1 beschriebene Filtern der GPS-Daten. Filtern bedeutet in diesem Zusammenhang, dass ein Teil der fehlerhaften Ausreißer aus den GSP-Spuren entfernt werden.

Aufbauend auf den Segmenten wird schließlich der Prozess der tatsächlichen Erkennung der Verkehrsmittel mit Hilfe eines Entscheidungsbaums und der berechneten Werte erklärt. Die aus dem Entscheidungsbaum gewonnenen Erkenntnisse werden schlussendlich ein letztes Mal überprüft, um sinnfreie bzw. sehr unwahrscheinliche Wechsel zwischen Verkehrsmitteln zu verhindern. 

Der vierte und letzte Abschnitt des Hauptteils befasst sich mit der Auswertung der gewonnenen Erkenntnisse aus den vorhergehenden Abschnitten sowie den Testläufen mit neuen GPS-Spuren mit und ohne zusätzliche GIS-Informationen. Dabei wird auch untersucht, wie sich die verschiedenen Zusatzinformationen wie z.B. die Geschwindigkeit oder die Stopprate auf die tatsächliche Erkennung auswirken.
