\chapter{Einleitung}
Der Mensch produziert täglich eine extrem große Menge an Daten. Allein auf Youtube werden pro Minute 300 Stunden Video-Material veröffentlicht und täglich hunderte Millionen Stunden von Videos konsumiert.  \cite{youtube_statistics_2015} Hochgerechnet auf das gesamte Internet und die gesamte Bevölkerung ergibt dies eine unvorstellbar große Menge an Daten die bewusst oder auch unbewusst generiert werden. 

Sehr viel an Daten wird auch durch diverse Fitnessgadgets, Smartwatches, Smartphones sowie Navigations-Geräten und Ähnlichem generiert. Abseits von Fitnesswerten sind all diese Geräte im Regelfall in der Lage GPS-Spuren aufzuzeichnen. Dies bedeutet, dass man genau nachvollziehen kann, wann man wo unterwegs war. Mit ein wenig Rechenarbeit kann man auch die Geschwindigkeit und viele andere Werte berechnen sofern dies die Geräte nicht schon selbst machen. 

Genau auf diesen GPS-Daten basiert diese Arbeit. Dabei ist es nicht wichtig von welcher Person diese Daten stammen sondern, dass sich mit Hilfe dieser aufgezeichneten Daten feststellen lässt, wann ein Individuum sich auf welcher Strecke mit welchem Verkehrsmittel fortbewegt hat.

Analysiert man viele dieser Daten so lassen sich viele Erkenntnisse daraus gewinnen. Unter anderem lassen sich zum Beispiel viel frequentierte Strecken in der Infrastruktur finden und mögliche Engstellen erkennen. Neben Engstellen lassen sich damit auch mögliche Verbesserungen, Einsparungen oder auch verstecktes Potential für zukünftige Projekte entdecken.

\section{Ziele dieser Arbeit}
Das Hauptziel dieser Arbeit ist es, einen Prototypen zu erstellen, welcher anhand von aufgezeichneten GPS-Spuren und in Kombination mit verschieden Methodiken das benutzte Verkehrsmittel mit einer möglichst hohen Wahrscheinlichkeit bestimmt. Diese aufgezeichneten GPS-Spuren enthalten dabei keinerlei Informationen über die jeweilige Person. Deshalb erfolgt die Auswertung ausschließlich über die jeweilige GPS-Spur sowie über öffentlich zugängliche Daten wie zum Beispiel Busstationen und Gleise. Jede dieser Aufzeichnung kann mehrere Verkehrsmittel beinhalten und von unterschiedlicher Länge und Dauer sein.

Die in dieser Arbeit berücksichtigten Verkehrsmittel sind in folgender Liste ersichtlich. Besonders interessant ist hierbei die Unterscheidung von Bus und Auto in verkehrsabhängigen Situationen bzw. in der Stadt da diese Transporttypen in diesen Situationen sehr ähnliche Verhalten und Werte aufweisen. 

\begin{itemize}
\item Fußgänger
\item Fahrrad
\item Bus
\item Auto (stellvertretend für alle nicht öffentliche, motorisierte Fahrzeuge)
\item Zug
\end{itemize}

Es soll weiters, für jede Person die ein Gerät besitzt das in der Lage ist eine GPS-Spur im GPX-Format aufzuzeichnen, möglich sein, diese Spur analysieren zu lassen. Dies bedeutet, dass keine speziellen Geräte oder andere Sensoren benötigt werden und dass sich dieser Prototyp mit möglichst geringen Anpassungen (Trainingsdaten, GIS-Daten und Grenzwerte in der Konfiguration) auch auf andere Regionen anwenden lässt. Zum Aufzeichnen der in dieser Arbeit verwendeten GPS-Spuren, wurde mehre Smartphones mit der App ``MyTrack`` sowie mehrere GPS-Geräte benutzt.

Im Zuge dieser Arbeit wird auch untersucht welches Level an Genauigkeit sowohl mit als auch ohne geografischen Zusatzinformationen im Raum Vorarlberg erreicht werden kann. Dabei soll es nach der Analyse die Möglichkeit geben, die automatisch bestimmten Verkehrsmittel manuell zu korrigieren sollten diese nicht mit der Realität übereinstimmen. Weiters sollen diese manuellen Änderungen in die Auswertung mit einfließen. 

Schlussendlich soll mit Hilfe der Auswertungen auch eine Aussage über die erzielte Genauigkeit mit und ohne den verwendeten Zusatzinformationen gemacht werden. Außerdem soll eine Aussage darüber getroffen werden können, ob noch weitere Zusatzdaten für eine noch genauere Bestimmung benötigt werden würden und welche Daten dies sein könnten. 

\subsubsection{Nichtziele}

In dieser Arbeit werden die diverse Schlussfolgerungsmodelle (neuronales Netz, Bayessches Netz, Random Forest, Support Vector Machine, ...) nicht betrachtet oder verglichen sondern es wird auf Modell gesetzt das bereits bei anderen Arbeiten wie z.B. \cite{stenneth_transportation_2011, reddy_using_2010, sebastian_nagel_moglichkeitsstudie_2011,zheng_learning_2008} vielversprechende Ergebnisse erreicht hat. Dieses Modell ist der Entscheidungsbaum. Außerdem werden die Daten zur Analyse bzw. Auswertung nicht in Echtzeit betrachtet sondern in Form einer GPS-Spur an den Prototypen übergeben.

Die Frage, an welcher Position man das Gerät zur Aufzeichnung am besten trägt um möglichst genaue GPS-Daten zu erhalten wird nicht weiter verfolgt da die Daten möglichst realistisch sein sollen. Auch bleibt die Frage nach dem Energierverbrauch der App bzw. wie eine möglichst energieschonende App und die dazugehörige Kommunikation aufgebaut sein könnte unberücksichtigt. Schlussendlich ist eine umfassende Behandlung der Themen Sicherheit und Privatsphäre im Rahmen der vorliegenden Arbeit nicht möglich und daher bleiben auch diese Themen unberührt. An dieser Stelle soll allerdings erwähnt werden, dass keine benutzerspezifischen Information in den GPS-Spuren benötigt oder vom Prototypen generiert oder gespeichert werden.

\section{Motivation und Nutzen}
Schon früher wurde versucht Aufzeichnungen über die Verkehrswege von verschiedenen Menschen zu sammeln. Aber die Protokolle in Papierform sowie die Telefonbefragungen waren zu aufwändig und die Menschen nicht zuverlässig genug. Darum ist es von entscheidendem Vorteil eine App oder ein Gerät zur Verfügung zu haben, welches die Vorgänge des Aufzeichnens möglichst genau für einen übernimmt. \cite{zheng_understanding_2010}

Wird eine Auswertung mit einer für das Zielgebiet aussagekräftigen Anzahl an Personen durchgeführt, so kann das Resultat  für verschiedenste Zwecke verwendet werden. Auch ohne spezielle Analyse kann rein durch die Betrachtung der gesammelten GPS-Spuren festgestellt werden, welche Routen besonders häufig benutzt werden und eventuelle Engpässe erkennen oder alternative Möglichkeiten entdeckt werden.

Zieht man nun verschiedene Werte aus der Auswertung hinzu kann auch festgestellt werden wo sich zum Beispiel verkehrstechnische Engstellen befinden und welche Routen sehr populär sind oder Aussagen über die allgemeine Verkehrssituation machen. Durch die gesammelten Daten könnten sich auch Simulationen für anstehende Bauvorhaben machen lassen und auch versucht werden eine Vorhersage für bestimmte Situationen zu tätigen. Diese Aspekte können unter anderem für das Verkehrsministerium,  den öffentlichen Personennahverkehr oder auch für die Stadtplanung sehr interessant sein (Optimierung von Auslastung, Einsparungspotentiale, Engpässe, ...).

Eine ganze Reihe von Apps lässt sich mit den Auswertungen erstellen. Diese Apps könnten die Auswertungen in soziale Medien zu integrieren, für Fitnessanalysen verwendet werden, einen einfachen Rückblick über die eigene Fortbewegung ermöglichen oder für Umweltbewusste errechnen wie viel CO2 sie produziert oder gespart haben. Ein Reisetagebuch könnte daraus genau so Nutzen ziehen wie eine App die beim Autofahren Auskunft über die aktuell billigste Tankstelle in näherer Umgebung gibt oder eine App die einfach nur Vorschläge für alternative, schnellere Routen zu einem bekannten Ziel anbietet.

Zusammenfassend kann man sagen, dass die Verwendungsmöglichkeiten für solche Daten umfangreich sind und sich am besten unter den Begriffen kontextorientierte, geographische Apps zusammenfassen lassen. Nicht zuletzt öffnen sich mit solchen Daten aber auch umfangreiche Möglichkeiten für die Werbebranche.

\section{State of the Art}
Zu den Meilensteinen auf diesem Gebiet der Forschung zählt sicher die Arbeit von Yu Zheng in welcher er unter anderem auf die Erkennung von den Abschnitten mit nur einem Verkehrsmittel eingeht. Weiters verwendete er in seiner Arbeit auch einen mit dem von den GPS-Spuren gesammelten geographischen Wissen aufgebauten Graphen welcher zur weiteren Auswertung verwendet wurde.\cite{zheng_understanding_2010}

Mit der Frage wie geographische Daten in eine solche Analyse miteinbezogen werden können hat sich auch Leon Stenneth beschäftigt. Dabei hat er nicht nur fixe Daten wie Gleise und Busstationen sondern auch aktuelle Buspositionen miteinbezogen. \cite{stenneth_transportation_2011}

Sowohl Stenneth als auch Zheng haben in ihren Arbeiten detailiert erklärt wieso sie welche Attribute (Geschwindigkeit, Beschleunigung, ...) für die Bestimmung des Verkehrsmittels verwendet haben und sie haben diese auch durch Versuche nach ihrer Wichtigkeit gereiht. Außerdem haben beide und auch Sasank Reddy mehrere Schlussfolgerungsmodelle (Entscheidungsbaum, Bayessches Netz, Markov Modelle, Random Forest, ...) betrachtet und miteinander verglichen. \cite{reddy_using_2010, stenneth_transportation_2011, zheng_understanding_2010}

Wie man mit Verbindungsabbrüchen umgeht und zwischen ähnlichen Verkehrsmitteln unterscheiden kann hat unter anderem auch Filip Biljecki beschäftigt. Des weiteren baut er für die unterschiedlichen Kategorien von Verkehrsmittel ein hierarchisches Modell auf, welches ihm helfen soll bessere Entscheidungen zu treffen. \cite{biljecki_transportation_2013}

\subsection{Daten}
Ein wesentlicher Unterschied zwischen all den betrachteten Publikationen sind die verwendeten Daten. Einzig die GPS-Spuren bilden eine gemeinsame Basis. Manche untersuchten die Verwendung von GSM- und WIFI-Informationen \cite{reddy_using_2010},  stützten sich auf Zusatzinformationen durch weitere Sensoren wie zum Beispiel ein Beschleunigungssensor \cite{reddy_using_2010, nadine_schussler_improving_2011}. Andere wie Leon Stenneth verwendeten Live-Informationen von den öffentlichen Verkehrsmittel und kombinierten diese mit GIS-Informationen, seien es Busstationen, Bahnstrecken, das Straßennetz oder Parkplätze \cite{stenneth_transportation_2011}.

\subsection{Verkehrsmittel}
Ein weiterer Unterschied zwischen den Publikationen sind die betrachteten Verkehrsmittel. Hierbei reicht die Spanne der unterschiedenen Verkehrsmittel von "Gehen und Motorisiert" (siehe \cite{reddy_using_2010}) bis hin zu "Gehen, Zug, U-Bahn, Rad, Auto, Straßenbahn, Bus, Fähre, Segelboot und Flugzeug"\ (siehe \cite{biljecki_transportation_2013}).

\section{Weiterer Aufbau der Arbeit}
Der Hauptteil der vorliegenden Arbeit gliedert sich in fünf große Abschnitte:

Im ersten wird auf die Akquirierung der GPS-Daten eingegangen. Dies umfasst sowohl die gesammelt GPS-Spuren und deren Struktur, sowie die verwendeten GIS-Daten. Dabei geht es einerseits um deren Herkunft als auch darum wie diese extrahiert wurden und wie auch diese Daten aufgebaut sind. Außerdem werden auch andere Daten wie zum Beispiel GPS-Daten von Bussen des ÖPNV in Betracht gezogen.

Der zweite Abschnitt behandelt den entwickelten Prototypen der die übergebenen GPS-Spuren analysiert. Dabei werden einerseits dessen Funktionalitäten erklärt sowie der grundlegende Ablauf für den Benutzer dargelegt. Weiters wird auch auf die Architektur des Prototyps sowie auf dessen Konfigurationsmöglichkeiten eingegangen.

Der dritte Abschnitt befasst sich sowohl mit dem Filtern als auch dem Aufbereiten der Daten sowie das Aufteilen von GPS-Spuren in Teile in denen nur ein Verkehrsmittel verwendet wurde. Filtern der Daten bedeutet in diesem Zusammenhang, dass Ausreißer aus den GSP-Spuren entfernt werden. Diese Ausreißer können aufgrund von Geschwindigkeitssprüngen als auch unwahrscheinlich große Sprünge im dreidimensionalen Raum sein. Mit Aufbereiten der Daten ist gemeint, dass Werte zu GPS-Punkten für die spätere Analyse berechnet werden. Diese Werte können sowohl die Geschwindigkeit, Beschleunigung, Distanz als auch der Abstand zu der nächsten Bushaltestelle sein. Schlussendlich sollen die gefilterten und erweiterten Daten verwendet werden um eine einfache und sichere Aufteilung der GPS-Spuren anhand der Verkehrsmittel zu ermöglichen.

Aufbauend auf die Resultate aus dem dritten Abschnitt befasst sich der vierte Abschnitt mit der tatsächlichen Erkennung der Verkehrsmittel anhand der berechneten Werte und den einzelnen Abschnitten der GPS-Spur. Die aus dem Entscheidungsbaum gewonnenen Erkenntnisse werden schlussendlich ein letztes Mal überprüft um sinnfreie bzw. sehr unwahrscheinliche Wechsel zwischen Verkehrsmitteln zu verhindern.

Im fünften und letzten Abschnitt des Hauptteils befasst sich mit der Auswertung der gewonnenen Erkenntnisse aus den vorhergehenden Abschnitten sowie den Testläufen mit neuen GPS-Spuren mit und ohne zusätzliche GIS-Informationen.
