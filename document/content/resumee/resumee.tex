\chapter{Ausblick}
Die Ergebnisse dieser Arbeit sind bereits im Abschnitt ``\ref{auswertung} \nameref{auswertung}`` detailliert beschrieben. Es kann jedoch hervorgehoben werden, dass die verwendeten Algorithmen und Methodiken die erwarteten Ergebnisse lieferten. Es ist weiters zu erwarten, dass die Trefferquote beim Klassifizieren der Segmente mit weiteren Trainingsdaten ansteigt. 

\section{GPX-Daten im Überblick}
Wie bereits erwähnt, konnte während der Arbeit an dem Prototypen festgestellt werden, dass die Qualität und Diversität der verwendeten GPX-Daten verbesserungswürdig ist und eine größere Anzahl an Trainingsdaten benötigt werden würde um ein besseres Resultat bei der Erkennung bzw. Klassifizierung zu erzielen. 

\subsection{GPX-Datenqualität}
Mit Qualität der GPX-Daten sind zwei verschiedene Dinge gemeint. Zum einen wäre es einfacher zu Filtern, wenn in den GPX-Daten ein Genauigkeitswert (z.B. errechnet über die Anzahl der momentan verfügbaren Satelliten) enthalten würden, wie es bereits in anderen Publikationen (\cite{stenneth_transportation_2011}, \cite{nadine_schussler_improving_2011}) gezeigt wurde. Dadurch könnte besser abgeschätzt werden ob die aktuellen Geschwindigkeitswerte auch im aktuellen Kontext realistisch sind und nicht nur im globalen Kontext. Sprich ist die Geschwindigkeit von mehr als 100 km/h realistisch, wenn die aufzeichnende Person gerade zu Fuß unterwegs ist oder handelt es sich eher um einen fehlerhaften Ausreißer der sich über einen niederen Genauigkeitswert erkennen lassen würde.

Der andere Punkt zum Thema Qualität der GPX-Daten ist jener, dass die Genauigkeit bei der händischen Segmentierung der Trainingsdaten sehr von Person zu Person variiert und es durchaus vorkommen kann, dass eine Zugfahrt laut GPX-Datei 10 GPX-Punkte länger dauert als sie eigentlich sollte. Dies kann vor allem bei kurzen Strecken wie sie in den aktuellen Testdaten oft vorkommen zu einer Beeinflussung von den Geschwindigkeits- und Beschleunigungswerten kommen kann. 

\subsection{Diversität der GPX-Daten}
Mitunter konnte festgestellt werden, dass bei den verwendeten GPX-Daten immer wieder dieselben Strecken mit den selben Verkehrsmitteln aufgezeichnet wurden. Wesentlich interessanter wäre es natürlich wenn es eine größere Anzahl von Teilnehmen bei dem Aufzeichnen der GPX-Tracks geben würde. Außerdem wäre es natürlich interessant wenn diese Teilnehmer möglichst unterschiedliche Fortbewegungsgewohnheiten haben. Damit ist gemeint, dass es durchaus öfter dieselbe Strecke in den Trainingsdaten geben darf, aber diese mit möglichst vielen verschiedenen Verkehrsmitteln bewältigt wird. Weiters wäre es auch beim Individualverkehr interessant wie sich verschieden Persönlichkeiten im Sinne eines aggressiven/geschwindigkeitsbetonten und passiven/gemütlichen Fahrstils auswirken. 

Ferner hat sich gezeigt, dass man die Testdaten möglichst auf das Hauptinteressengebiet beschränken soll. Damit ist gemeint, dass das Aufnehmen einer MTB-Tour in die Trainingsdaten nur bedingt interessant bzw. hilfreich ist. Denn früher oder später bringen die entweder sehr hohen (abwärts) oder sehr niederen (aufwärts) Geschwindigkeiten unnötige Irritationen in den Klassifizierungsprozess. Ambitionierte Sportler liefern keine repräsentativen Daten um eine Analyse für den täglichen Verkehrsteilnehmer. 

\subsection{Quantität der GPX-Daten}
Neben qualitativen Verbesserungen, ist es auch praktisch, wenn man auf eine große Datenmenge zurückgreifen kann. Zumal Abhängig vom gewählten Algorithmus für die Generierung des Entscheidungsbaums eine große Datenmenge benötigt wird um ein repräsentatives Ergebnis zu erhalten (sieh Abschnitt \ref{entscheidungsbaumAlgorithmen} \nameref{entscheidungsbaumAlgorithmen}).

\section{GIS-Daten}
Wie bereits am Beginn der Arbeit erwähnt, gab es keine Möglichkeit auf die Daten der öffentlichen Verkehrsmittel zuzugreifen. Allerdings konnte bereits Stenneth \cite{stenneth_transportation_2011} sehr genaue Resultate mit diesen Daten zu erzielen. Deshalb kann angenommen werden, dass auch für diesen Prototyp die Genauigkeit erhöht werden kann, wenn auf solche Daten zurückgegriffen werden kann. Dies könnte auch dann von großem Vorteil sein, wenn man zwischen verschieden öffentlichen Verkehrsmitteln wie zum Beispiel U-Bahn, Straßenbahn sowie Bus unterscheiden möchte. 

\section{Entscheidungsbaum}
Zu der Thematik der Entscheidungsbäume sollte erwähnt werden, dass die Möglichkeit der automatisierten Aktualisierung bzw. die erneute Generierung der Entscheidungsbäume eine erhebliche Vereinfachung beim Einfügen von neuen Trainingsdaten darstellen würde. Die Verarbeitung der Trainingsdaten kann bereits mit einem Befehlt gestartet werden. Jedoch muss weiterhin die erzeugte Trainingsdatendatei in RapidMiner importiert, der Entscheidungsbaum neu generiert und schlussendlich der Baum in Textform in das Projekt kopiert werden (sieh \nameref{anhang2}). Das Parsen den Entscheidungsbaumes wird dann wiederum automatisch vom Prototypen gemacht (sieh Abschnitt \ref{parsenCachenEntscheidungsbaum} \nameref{parsenCachenEntscheidungsbaum}). 

Die Schritte im RapidMiner sind mit der Community-Version leider unumgänglich. Allerdings gibt es auch eine Serverversion von RapidMiner bei dem es laut Dokumentation verschiedene Schnittstellen wie z.B. via Webservices möglich sein soll zu Kommunizieren. 

\section{Lernfähigkeit des Systems}
Nicht zuletzt wäre es für das Gesamtsystem natürlich sehr interessant, wenn aus den von den Benutzern und Benutzerinnen korrigierten Entscheidungen gelernt werden könnte und diese Korrekturen in zukünftige Klassifizierungen miteinfließen würden. 
