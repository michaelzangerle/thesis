\section*{Zusammenfassung}

Die zentrale Fragestellung dieser Arbeit ist die Suche nach einer Methode um GPS-Tracks  aus dem Großraum Vorarlberg hinsichtlich der verwendeten Verkehrsmitteln zu analysieren. Dafür sollen die Tracks selbst über keine zusätzlichen Informationen verfügen und auch keine zusätzlichen Sensoren verwendet werden.

Für die eigentliche Analyse wurde eine grundlegende Filterlogik implementiert um fehlerhafte Ausreißer aus den Rohdaten zu entfernen. Anschließend wurden die verbleibenden Wegpunkte segmentiert um sie in Folge in Segmente zu Gruppieren die mit nur einem Verkehrsmittel bewältigt worden sind. Für diese Segmente wurden Zusatzwerte (z.B. Stopprate, Geschwindigkeit, ...) berechnet um darüber das Verkehrsmittel bestimmen zu können. Für die Bestimmung wird auf Entscheidungsbäume als Schlussfolgerungsmodell gesetzt da mit diesem in verschiedenen ähnlichen Publikationen gute Ergebnisse erzielt worden sind. Schlussendlich werden die Resultate ein letztes Mal mit Hilfe des Kontexts auf Plausibilität und Sinnhaftigkeit untersucht und gegebenenfalls angepasst.

Das Resultat dieser Arbeit ist der Prototyp einer Webapplikation welche GPS-Spuren auf die verwendeten Verkehrsmittel untersucht und diese anhand von verschiedenen Analysemethode bestimmt. Für diesen Prototyp wurde eine Analysemethode basierend auf Geschwindigkeits- und Beschleunigungswerten und eine zweite Analysemethode die zusätzlich GIS-Daten verwendet, implementiert. Es konnte eine Erkennungsrate von bis zu 66\% bzw. 75\% erreicht werden. Jedoch muss hinzugefügt werden, dass nur wenig Trainingsdaten zur Verfügung standen. Eines der wichtigsten Dinge sind aber diese Daten. So sollten Test- und Trainingsdaten in ausreichender Quantität als auch Qualität vorhanden sein um eine maximale Aussagekraft und eine bestmögliche Erkennungsrate zu erreichen. Dies gilt sowohl für das Trainieren des Schlussfolgerungsmodells als auch für die schlussendliche Erkennung.
\clearpage

\section*{Abstract}

\todo{Abstract ergänzen}