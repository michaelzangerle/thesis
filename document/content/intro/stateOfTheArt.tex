\chapter{Stand der Technik}
Zu den Meilensteinen auf dem Forschungsgebiet der Verkehrsmittelerkennung (Transport Mode Detection) zählt die Arbeit von Yu Zheng, in welcher er unter anderem auf die Erkennung von den Abschnitten mit nur einem Verkehrsmittel eingeht. Weiters verwendete er in seiner Arbeit auch einen mit dem von den GPS-Spuren gesammelten geographischen Wissen aufgebauten Graphen, welcher zur abschließenden Korrektur verwendet wurde.\cite{zheng_understanding_2010} \cite{zheng_understanding_2008} \cite{zheng_learning_2008}

Mit der Frage, wie geographische Daten in eine solche Analyse miteinbezogen werden können hat sich unter anderem Leon Stenneth beschäftigt. Dabei hat er nicht nur fixe Daten wie Gleise und Busstationen, sondern auch aktuelle Buspositionen miteinbezogen. \cite{stenneth_transportation_2011}

Sowohl Stenneth als auch Zheng haben in ihren Arbeiten detailliert erklärt, wieso sie welche Attribute (Geschwindigkeit, Beschleunigung, ...) für die Bestimmung des Verkehrsmittels verwendet haben und sie haben diese auch durch Versuche nach ihrer Wichtigkeit gereiht. Außerdem haben beide und auch Sasank Reddy \cite{reddy_determining_2008} mehrere Schlussfolgerungsmodelle (Entscheidungsbaum, Bayessches Netz, Markov Modelle, Random Forest, ...) betrachtet und miteinander verglichen. 

Wie man mit Verbindungsabbrüchen umgeht und zwischen ähnlichen Verkehrsmitteln unterscheiden kann, hat unter anderem auch Filip Biljecki beschäftigt. Des Weiteren baut er für die unterschiedlichen Kategorien von Verkehrsmitteln ein hierarchisches Modell auf, welches ihm helfen soll, bessere Entscheidungen zu treffen. \cite{biljecki_transportation_2013}

\section{Verwendete Daten}
Ein wesentlicher Unterschied zwischen all den betrachteten Publikationen sind die verwendeten Daten. Nur die GPS-Spuren bilden eine gemeinsame Basis. Manche Autoren untersuchten die Verwendung von GSM- und Wifi-Informationen \cite{reddy_using_2010} oder stützten sich auf Zusatzinformationen durch weitere Sensoren wie zum Beispiel einem Beschleunigungssensor \cite{reddy_using_2010}  \cite{nadine_schussler_improving_2011}. Andere, wie Leon Stenneth, verwendeten Echtzeitinformationen von öffentlichen Verkehrsmitteln und kombinierten diese mit GIS-Informationen, seien es Busstationen, Bahnstrecken, das Straßennetz oder Parkplätze \cite{stenneth_transportation_2011}.

\section{Betrachtete Verkehrsmittel}
Ein weiterer Unterschied zwischen den Publikationen sind die betrachteten Verkehrsmittel. Hierbei reicht die Spanne der unterschiedenen Fortbewegungsmöglichkeiten von ``gehen, laufen, Fahrrad und motorisiert`` \cite{reddy_using_2010} bis hin zu ``gehen, Zug, U-Bahn, Rad, Auto, Straßenbahn, Bus, Fähre, Segelboot und Flugzeug`` \cite{biljecki_transportation_2013}.

\section{Analyse der Publikationen von Yu Zheng}
Zu den Meilensteinen auf dem Gebiet der ``Transport Mode Detection`` zählen die Publikationen von Yu Zheng und seinem Team: ``Understanding Mobility Based on GPS Data`` \cite{zheng_understanding_2008}, ``Understanding Transportation Modes Based on GPS Data for Web Applications`` \cite{zheng_understanding_2010} und ``Learning Transportation Mode from Raw GPS Data for Geographic Application on the Web`` \cite{zheng_learning_2008}. In diesen Artikeln wird unter anderem auf die Erkennung von Abschnitten mit nur einem Verkehrsmittel eingegangen. Weiters wird erklärt, wie in den Projekten ein aus analysierten GPS-Spuren erstellter Graph mit geographischem Wissen zur Verbesserung der Erkennungsrate beiträgt und mit welchen Schlussfolgerungsmodellen welche Ergebnisse erzielt werden konnten. 

\subsection{Geolife}
Die Applikation, welche im Zusammenhang mit den oben genannten Arbeiten entstanden ist, nennt sich ``Geolife`` und ist eine Webapplikation mit den Ansätzen eines sozialen Netzwerks. Dabei ging es darum, dass Benutzer und Benutzerinnen GPS-Spuren auf die Webseite laden konnten, diese Spur vom Algorithmus analysiert und die Verkehrsmittel bestimmt wurden. Dargestellt wurden die Resultate auf einer Karte, welche Benutzer und Benutzerinnen mit anderen teilen konnten.

Mit Hilfe der so gesammelten Daten konnte unter anderem Datamining betrieben und populäre Strecken festgestellt sowie Verkehrssituationen beurteilt werden. Außerdem konnten auch aktuelle Positionswerte mit einem GPS-fähigen Smartphone/Handy ausgewertet werden und Informationen wie z.B. Abfahrtszeiten der öffentlichen Verkehrsmittel angeboten werden. Wurde aber z.B. eine Strecke mit dem Auto in die Stadt gesucht, so konnte auf Grund der bereits analysierten Fahrten über die durchschnittliche Geschwindigkeit festgestellt werden, welche Strecke am schnellsten ans Ziel führt.

In diesem von Microsoft Research geführten Projekt standen umfangreiche Test- und Trainingsdaten (1,2 Millionen Kilometer und 48.000 Stunden) zur Verfügung. \cite{microsoft_research_geolife_2015} Ein Bildschirmfoto der Applikation ist in Abbildung \imgref{geolife} zu sehen. 

\img{0.9}{document/graphics/geolife.jpg}{Geolife (Quelle: research.microsoft.com)}{geolife}

\subsection{Segmentierung}
Die Publikationen von Zheng beinhalten auch eine der detailliertesten Beschreibungen des Segmentierungsvorgangs. Dabei werden GPS-Spuren in Abschnitte, in welchen nur ein Verkehrsmittel verwendet wird, unterteilt. Diese Abschnitte können dann in weiterer Folge genauer analysiert und das benutzte Verkehrsmittel bestimmt werden.

Beim Segmentieren stützt sich Zheng darauf, dass Personen bei einem Wechsel des Verkehrsmittels stehen bleiben und sich ein Stück zu Fuß bewegen. Dies bedeutet, dass es einige GPS-Punkte mit einer Geschwindigkeit von genau oder beinahe 0 km/h gibt. Diese aufeinanderfolgenden Punkte können dann in einem ``Gehen``-Segment zusammengefasst werden. Alle anderen Segmente werden vorläufig als ``Nicht-Geh``-Segmente klassifiziert. Zheng sagt außerdem, dass der Beginn und das Ende eines Segments mit dem Typ ``Gehen`` ein wichtiger Indikator für einen Wechsel des Fortbewegungsmittels ist. Diese Aussage stützt sich auf die Erkenntnisse aus GPS-Daten, die von 65 Personen über 10 Monate gesammelt wurden.

\subsection{Schlussfolgerungsmodelle} \label{schlussfolgerungsmodelle}
Um die Typen der ``Nicht-Geh-Segmente`` bestimmen zu können, hat sich Zheng auf verschiedene Zusatzinformationen basierend auf den Segmenten gestützt, darunter:

\begin{pitemize}
\item Distanz
\item maximale Geschwindigkeit
\item maximale Beschleunigung
\item durchschnittliche Beschleunigung
\item Richtungswechsel
\item Stopprate
\item erwartete Geschwindigkeit
\item Geschwindigkeitsänderungsrate
\end{pitemize}

In weiterer Folge stellte er fest, dass Stopp-, Richtungswechsel- und Geschwindigkeitsänderungsrate am effektivsten und stabilsten gegenüber den verschiedenen Verkehrssituationen sind. Diese drei Eigenschaften können auch mit ein paar der anderen kombiniert werden, um noch weitere Verbesserungen zu erhalten. Wurden allerdings zu viele Eigenschaften miteinbezogen, so konnte er eine Verringerung der Genauigkeit beim Bestimmen des Typs feststellen. 

Für die tatsächliche Bestimmung des Typs führte Zheng verschiedene Experimente mit dem Entscheidungsbaum, der Support Vector Machine, dem bayesschen Netz und dem Conditional Random Field durch. Dabei stellte er fest, dass der Entscheidungsbaum die besten Ergebnisse im Zusammenspiel mit der in Abschnitt ``\ref{schlussfolgerungsmodelle} \nameref{schlussfolgerungsmodelle}`` beschriebenen Segmentierungsmethode liefert. 

Während des Analysierens der Trainingsdaten wurde im Hintergrund ein Graph aufgebaut, welcher Informationen über die jeweils betrachtete geografische Region widerspiegelt. In diesen geografischen Regionen waren 28 große Städte in China sowie mehrere Städte in den USA, Südkorea und Japan enthalten. Mit diesem Graph wurden die Schlussfolgerungen nochmals überprüft, und es konnte eine weitere Verbesserung erzielt werden. Schlussendlich konnte der von Zheng entwickelte Algorithmus 76,2\% der Verkehrsmittel ohne jegliche Zusatzinformationen im Sinne von GIS-Daten oder anderen Sensordaten korrekt identifizieren.

\section{Analyse der Publikationen von Leon Stenneth}
Leon Stenneth vergleicht in der Publikation ``Transportation Mode Detection using Mobile Phones and GIS Information`` \cite{stenneth_transportation_2011}, ähnlich wie Zheng auch, verschiedene Schlussfolgerungsmodelle. Allerdings verwendete er auch zusätzliche GIS-Informationen, um eine genauere Bestimmung zu ermöglichen. Außerdem arbeitet die entwickelte Applikation nicht mit ganzen GPS-Spuren, sondern analysiert immer zwei Punkte innerhalb eines 30-Sekunden-Intervalls, wodurch auch das eigentliche Segmentieren entfällt.

\subsection{GIS-Informationen}
Die von Stenneth verwendeten GIS-Informationen beinhalten sowohl die Gleise von Zügen als auch Bushaltestellen sowie die aktuelle Position von allen Bussen in Chicago. Daraus berechnete er verschiedene Zusatzinformationen für die Bestimmung des Transporttyps:
\begin{pitemize}
\item durchschnittliche Nähe zu den Gleisen
\item durchschnittliche Nähe zu Bushaltestellen
\item durchschnittliche Nähe zum nächsten Bus
\end{pitemize}

\subsection{Schlussfolgerungsmodelle}
Um den Typ des jeweils betrachteten Abschnitts zu bestimmen, zieht auch Stenneth mehrere Zusatzwerte als Kriterien zu Hilfe:

\begin{pitemize}
\item durchschnittliche Geschwindigkeit
\item durchschnittliche Nähe zu den Gleisen
\item durchschnittlicher Abstand zu einem Bus
\item durchschnittliche Beschleunigung
\item durchschnittlicher Richtungswechsel
\item durchschnittliche Bushaltestellennähe
\item durchschnittliche Genauigkeit der Koordinaten
\item Distanz zum nächsten Bus
\end{pitemize}

Wie Zheng stellt auch Stenneth fest, dass nur ein paar der Zusatzwerte wirklich effektiv sind. In seinem Fall waren das die durchschnittliche Geschwindigkeit und Beschleunigung, die Nähe zu den Gleisen, der Abstand zu anderen Bussen sowie die Distanz zum nächsten Bus.

Die von Stenneth betrachteten Modelle sind Naive Bayes, bayessches Netz, Entscheidungsbaum, Random Forest und Multilayer Perceptron. In den Experimenten stellte er fest, dass der Random Forest mit den GIS-Informationen das vielversprechendste Modell mit mehr als 93\% richtig erkannten Verkehrsmitteln ist. Ohne GIS-Informationen schneidet auch der Random Forest ähnlich wie bei Zheng der Entscheidungsbaum mit 76\% ab. Mit GIS-Informationen erreicht auch der Entscheidungsbaum eine Erkennungsrate von mehr als 92\%. 

\section{Analyse der Publikationen von Filip Biljecki}
Filip Biljecki veröffentlichte eine Arbeit mit dem Titel ``Transportation mode-based segmentation and classification of movment trajectories`` \cite{biljecki_transportation_2013}. Darin vergleicht er nicht nur eine Vielzahl von Publikationen zu dem Thema Verkehrsmittelerkennung (unterer anderem \cite{schuessler_processing_2009}, \cite{zheng_understanding_2010}, \cite{reddy_using_2010}, \cite{gonzalez_automating_2010}) anhand der Fortbewegungsmittel sowie der Zusatzwerte, ob GIS-Daten verwendet wurden und welches Resultat erzielt worden ist, sondern führt auch ein hierarchisches Modell für die Erkennung der Transportmittel ein. Weiters verwendet er zur Bestimmung des Transportmittels auch GIS-Daten wie Bus- und Straßenbahnhaltestellen.

Ein Kapitel in dieser Arbeit ist auch der Behandlung von Störungen oder Unterbrechungen des GPS-Signals gewidmet. Darin wird aufgeführt, welche Fälle weiter behandelt werden können und welche sich mit diesem Ansatz nicht beheben lassen. In dieser Arbeit konnte eine Genauigkeit von 95\% erreicht werden.

\subsection{Segmentierung}
Für die Segmentierung verwendet Biljecki dieselbe Methodik wie Zheng, aber erweitert diese dahingehend, dass auch dann segmentiert wird, wenn ein Signalverlust (30 Sekunden keine neuen Werte) feststellt werden kann. Diese Entscheidung wird damit begründet, dass es bei einem Verkehrsmittelwechsel oft zu einem Signalverlust kommt. 
Überschreitet ein Stopp eine bestimmte Dauer (12 Sekunden) so wird auch segmentiert, denn dies könnte auch auf einen Wechsel hindeuten. Da die Track-Punkte nicht 100\% genau sind, wird alles unter 2km/h als Stopp eingestuft. Weil nach der Erkennung alle aufeinanderfolgenden Segmente vom selben Typ zusammengefasst werden, ist eine Übersegmentierung kein Problem.

\subsection{Schlussfolgerung}
Um das Fortbewegungsmittel feststellen zu können, verwendet Biljecki ein Expertensystem. Dieses System basiert auf einer Fuzzy-Logic und klassifiziert die Segmente mit Wahrscheinlichkeitswerten. Dazu verwendet dieses System auch die hierarchische Gliederung der verschiedenen Verkehrsmittel, welche in Abbildung \imgref{transportationModeHierarchy} ersichtlich ist. Diese Gliederung soll verhindern, dass sich das System zu früh auf einen Typ festlegen muss - das System kann den Typ dadurch schrittweise bestimmen.

\img{1}{document/graphics/transportationModeHierarchy.png}{Fortbewegungsmittel-Hierarchie (Quelle:  \cite{biljecki_transportation_2013})}{transportationModeHierarchy}

Zusammenfassend sagt Biljecki, dass es sehr schwierig ist, alle möglichen Fälle der Realität abzubilden und wirklich zufriedenstellende Resultate zu erhalten. Dies betrifft vor allem den Segmentierungs- und Bestimmungsprozess von Daten mit sehr vielen fehlerhaften Ausreißern (Rauschen) oder jene Fällen, in denen wenige Beispieldaten vorhanden sind. Weiters meint er, dass Fehler nicht zwangsläufig auf das System zurückzuführen sind, sondern dass es sich bei diesen Fehlern oft um spezielle Situationen handelt, welche sehr kompliziert zu modellieren sind oder drastische Auswirkungen auf die allgemeine Performanz haben.

Alles in allem kann aber gesagt werden, dass das von Biljecki vorgestellte Modell zehn verschiedene Verkehrsmittel unterscheiden kann. Dies sind mehr als in allen anderen vom Autoren betrachteten Publikationen. Außerdem konnte durch die Verwendung von GIS-Daten bessere Resultate (95\% bis 100\% je nach Daten) als in vielen anderen Publikationen im Bereich der Verkehrsmittelerkennung erzielt werden. 

\section{Analyse der Publikationen von Sasank Reddy}
Sasank Reddy hat sich in den Publikationen ``Using Mobile Phones to Determine Transportation Modes`` \cite{reddy_using_2010} und ``Determining Transportation Mode on Mobile Phones`` \cite{reddy_determining_2008} wie auch Stenneth und Zheng mit verschiedenen Schlussfolgerungsmodellen befasst. Bevor diese Modelle aber zum Einsatz kamen, wurde evaluiert, mit welchen weiteren Sensoren man sinnvolle Daten aufzeichnen könnte. Im Zuge dessen wurde auch überprüft, an welcher Stelle am Körper das Smartphone die genauesten Daten liefert und wie möglichst energieeffizient Daten aufgezeichnet und übermittelt werden können. Im Gegensatz zu anderen Publikationen wurde in diesen Arbeiten nicht genauer zwischen den motorisierten Verkehrsmitteln unterschieden.

\subsection{Sensoren}
Da die meisten Smartphones nicht nur über ein GPS-Modul sondern auch über WIFI, einen Beschleunigungssensor sowie Bluetooth und natürlich über ein GSM-Modul verfügen, wurde in diesen Arbeiten auch evaluiert, ob es möglich und sinnvoll ist, Daten von diesen Geräten und Sensoren miteinzubeziehen. 

Bluetooth  wäre zwar interessant, aber es kommt hauptsächlich innerhalb von Gebäuden zum Einsatz (TV, Radio, Computer, ...). Darum konnte dieser Sensor nicht für die Bestimmung des Verkehrsmittels eingesetzt werden. 

Wifi und GSM in den Erkennungsprozess miteinzubeziehen konnte nach einer Reihe von Versuchen ausgeschlossen werden, da der Grad der Verbesserung nur 0,6\% betrug und ein wesentlich höherer Energiebedarf gegeben war. Außerdem war die Abhängigkeit von Wifi und GSM in ländlichen Gegenden mit schlechtem Empfang und/oder wenig Wifis ein weiterer Grund, der gegen diese Sensoren sprach.

Die vielversprechendste Kombination war jene aus GPS-Daten und den Daten des Beschleunigungssensors, welche auch  für die weiteren Experimente verwendet wurde.

\subsection{Schlussfolgerungsmodelle}
Um das Fortbewegungsmittel des jeweiligen Abschnittes zu erkennen, zieht Reddy folgende Zusatzinformationen heran:

\begin{pitemize}
\item Geschwindigkeit
\item Varianz des Beschleunigungssensorsignals
\item 3 Beschleunigungssensorwerte (3Hz, 2Hz, 1Hz)
\end{pitemize}

Die in dieser Publikation betrachteten Schlussfolgerungsmodelle sind der Entscheidungsbaum, K-Means Clustering, Naive Bayes, Nearest Neighbor, Support Vector Machine sowie ein Continuous Hidden Markov Model und ein Entscheidungsbaum in Kombination mit einem Discrete Hidden Markov Model. Dabei konnte festgestellt werden, dass die letzte Variante die besten Resultate mit 93,6\% erzielte. Wie aber auch in den anderen Publikationen war der Entscheidungsbaum mit 91,3\% nicht weit abgeschlagen hinter dem kombinierten Ansatz.

\section{Zusammenfassung}

Neben den vorgestellten Publikationen zum Thema Verkehrsmittelerkennung gibt es noch weitere interessante Publikationen, deren Erfahrungen zum Teil in diese Arbeit eingeflossen sind. Unter diesen Publikationen sind unter anderem ``Processing raw data from global positioning systems without additional information`` \cite{schuessler_processing_2009}  und  ``Improving post-processing routines for gps oversavations using propted-recall data`` \cite{nadine_schussler_improving_2011} von Nadine Schüssler sowie ``Automating mode detection for travel behaviour analysis by using global positioning systems-enabled mobile phones and neural networks`` \cite{gonzalez_automating_2010} von Paola Gonzalez.

Für diese Arbeit wurde für den Segmentierungsprozess die Vorgehensweise von Yu Zheng \cite{zheng_understanding_2010} verwendet. Darin sind aber auch die von Filip Biljecki \cite{biljecki_transportation_2013} vorgeschlagenen Änderungen eingeflossen, welche eine weitere Segmentierungsregel bei Empfangsverlust beinhalten. 

Für die weitere Verarbeitung der GPS-Daten wurde der Entscheidungsbaum aufgrund des guten Abschneidens in den erwähnten Publikationen verwendet. Der Entscheidungsbaum wurde zur nachfolgenden Analyse und Auswertung der verbesserten Verkehrsmittelerkennung in zwei Varianten implementiert. In der einen Variante verwendet er nur aus den GPS-Daten berechnete Werte (Geschwindigkeit, Beschleunigung, ...), wie es auch in vielen anderen Arbeiten gemacht wurde. In der zweiten Variante greift der Entscheidungsbaum aber auch auf GIS-Informationen zurück, wie es zum Beispiel Leon Stenneth in seiner Publikation  \cite{stenneth_transportation_2011} beschrieben hat.

Ein weiterer Grund für die Verwendung von GIS-Daten war neben dem guten Abschneiden der GIS-Daten-gestützten Verkehrsmittelerkennung, die eigene Zielsetzung, dass keine zusätzlichen Sensoren oder speziellen Geräten neben dem Gerät für die Aufzeichnung der GPS-Daten verwendet werden sollen.
