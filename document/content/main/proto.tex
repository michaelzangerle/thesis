\chapter{Der Prototyp}

Im Rahmen dieser Thesis  wurde ein Prototyp zur Erkennung von Verkehrsmitteln auf Basis von GPS-Daten entwickelt. Im Prinzip sollte jede Person, die ein Gerät zur Aufzeichnung von GPS-Tracks sowie einen Computer mit Internetzugang besitzt, ihre Daten mit diesem Prototypen analysieren können. In Abstimmung mit diesem Ziel wurde eine Webapplikation entwickelt. Diese Applikation verwendet serverseitig das PHP-Framework Symfony als Basis und stellt eine REST-Schnittstelle zur Kommunikation zur Verfügung. Für die Persistierung wurde das Doctrine-ORM in Verbindung mit einer mySQL-Datenbank verwendet.

Der Prototyp bietet neben dem Filtern, Segmentieren und Analysieren der GPS-Daten auch eine Visualisierung der analysierten Daten an. Dies bedeutet, dass das Resultat der Verarbeitung auf einer Karte dargestellt wird. Zusätzlich zur Darstellung der Karte wird dem Benutzer/der Benutzerin dadurch auch die Möglichkeit geboten, die Resultate des Prozesses zu bearbeiten und somit gegebenenfalls das Verkehrsmittel zu korrigieren. 

Sowohl der vom System bestimmte als auch der vom Benutzer/von der Benutzerin korrigierte Verkehrsmitteltyp werden in der Datenbank abgelegt, um Aussagen über die Genauigkeit beider Verfahren (mit und ohne GIS-Daten) treffen zu können. Diese Auswertungen werden wiederum durch verschiedene Diagramme und Statistiken visualisiert.

\section{Aufbau und Architektur}
Der Kern der Webapplikation ist in einer Pipes-and-Filter Architektur aufgebaut. Diese Architektur bietet sich an, wenn man einen Datenstrom verarbeiten will. Dabei werden die einzelnen Schritte der Verarbeitung in sogenannte Filter unterteilt, und diese werden mit Kanälen (Pipes) verbunden. Die Daten fließen somit in einen Filter, werden dort bearbeitet und schlussendlich über die verbundenen Kanäle zu einem oder mehreren nachfolgenden Filtern weitergeleitet. Am Ende einer solchen Kette steht eine Senke (sink), in welcher die Daten dann abgelegt werden (z.B. eine Datei). Durch dieses Architekturmuster wird eine Codegliederung in einzelne Komponenten/Arbeitsschritte in gewisser Weise bereits vorgegeben. Aber es ergeben sich auch andere Vorteile, wie z.B. Flexibilität bezüglich der Datenquelle, Austauschbarkeit von einzelnen Filterimplementierungen oder auch der Repräsentation und Persistierung der Endergebnisse. Weiters müssen bei diesem Ansatz auch keine Zwischendateien o.Ä. geschrieben werden. \cite{buschmann_pipes_1998}

Diese Flexibilität, welche es ermöglicht, Filter bis zu einem gewissen Grad zu kombinieren und in verschiedenen Reihenfolgen zu verbinden, wurde für den Prototyp benötigt, um z.B. beide Analysemethoden möglichst einfach abbilden zu können. Außerdem werden im Falle dieses Prototyps die Trainingsdaten in eine Datei geschrieben. Bei einer Anfrage über die Webapplikation werden die Daten in der Datenbank abgelegt. Hierbei unterscheiden sich nicht nur die Persistierungsarten sondern auch die Darstellung der Daten selbst, da in die Datei im CSV-Format geschrieben wird. Durch diesen Ansatz konnten viele Komponenten in anderen Benutzungsfällen wiederverwendet werden.

\subsection{Pipes-and-Filter-Architektur für Trainingsdaten}
Die Anordnung der einzelnen Filter für die Verarbeitung der GPS-Daten ist im Falle der Trainingsdaten in Abbildung \imgref{pipesAndFilterTraining} ersichtlich. Hierbei wird das Verzeichnis mit den Trainingsdaten definiert und dem ersten Filter (FileReader) werden nach und nach die einzelnen Dateien übergeben. Nachdem die Daten den Trackpoint- und Tracksegmentfilter (grün ohne GIS-Daten, orange mit GIS-Daten) durchlaufen haben, werden sie schließlich von der letzten Komponente (FileWriter) in eine Datei geschrieben.

\img{1}{document/graphics/pipes_and_filter_diagram_training.png}{Pipes- und Filter- Struktur für Trainingsdaten}{pipesAndFilterTraining}

\subsubsection{FileReader-Filter}
Der FileReader-Filter liest die Datei mit dem angegebenen Dateinamen als XML-Datei ein, sucht sich alle Segmente (trkseg) aus dieser Datei und gibt diese schließlich an den nächsten Filter weiter. Als eingelesene Datei erwartet sich dieser Filter eine XML-Datei im GPX-Format.

\subsubsection{Trackpoint-Filter}
Der Trackpoint-Filter ist nicht nur ein Filter im Sinne des Architekturmusters sondern auch im Sinne seiner Aufgabe. Er versucht all jene Trackpunkte (trkpt) herauszufiltern, die nicht den konfigurierten Grenzwerten entsprechen. Dies bedeutet, dass all jene Trackpunkte gefiltert werden, bei welchen der Zeitabstand, die Geschwindigkeit oder die Änderung im Bereich der Höhenmeter unter/über den minimalen/maximalen Grenzwerten liegen. Dadurch werden die einzelnen GPX-Segmente von den gröbsten fehlerhaften Ausreißern bereinigt, bevor sie an den nächsten Filter weitergegeben werden.

\subsubsection{Tracksegment-Filter}
\label{tracksegmentFilter}
Der Tracksegment-Filter ist einer der wichtigsten Filter im Verarbeitungsprozess. Er berechnet für jedes Segment die für die weitere Verarbeitung benötigten Zusatzwerte. Dies bedeutet, dass er die mittlere und maximale Beschleunigung, die mittlere und maximale Geschwindigkeit sowie die Stopprate berechnet. Weitere Informationen über die gewählten Zusatzwerte sowie die Auswahl selbst sind im Abschnitt ``\ref{schlussfolgerungsvariablen} \nameref{schlussfolgerungsvariablen}`` zu finden.  Im Falle der Trainingsdaten ergänzt er die Segmentdaten noch mit dem in den Trainingsdaten definierten Verkehrsmitteltyp.

\subsubsection{GISTracksegment-Filter}
\label{gisTracksegmentFilter}
Der GISTracksegment-Filter macht im Wesentlichen dasselbe wie der normale Trackseg- ment-Filter, allerdings berechnet er zusätzlich zu den Geschwindigkeits- und Beschleunigungswerten noch die Abstände zu den verschiedenen Infrastrukturen wie Busstationen, Schienen und Autobahnen. Diese Werte werden bei der Analyse mit GIS-Daten benötigt.

\subsubsection{FileWriter}
Der FileWriter übernimmt das Schreiben der übergebenen Daten in eine Datei. Hierbei werden die vom Tracksegment-Filter übergebenen Segmente in einer Datei im CSV-Format abgelegt, welche in weiterer Folge im RapidMiner verwendet wird.

\subsection{Pipes und Filter-Architektur für die Webapplikation}
Die Anordnung der einzelnen Filter für die Verarbeitung der GPS-Daten, die über die Webapplikation verarbeitet werden sollen, ist in Abbildung \imgref{pipesAndFilter} ersichtlich. 

Der Ablauf ist hierbei ähnlich wie jener bei den Trainingsdaten, und es wird wiederum zwischen den zwei Modi mit GIS-Daten (orange) oder ohne GIS-Daten (grün) unterschieden. Ausgehend vom FileReader und Trackpoint-Filter kommen die Daten in den Segmentation-Filter. Von dort kommen sie, je nach Analyse-Art, in den Tracksegment-Filter oder in den GISTracksegment-Filter. Das Ergebnis dieser Filter kommt danach in den TravelModel-Filter, wo die eigentlichen Verkehrsmittel bestimmt werden. Anschließend werden die Ergebnisse dem Postprocess-Filter übergeben, welcher die bestimmten Verkehrsmittel ein letztes Mal überprüft. Nun werden sie an den Database-Filter übergeben, welcher die Ergebnisse für die Datenbank aufbereitet und in dieser ablegt.

\img{0.85}{document/graphics/pipes_and_filter_diagram.png}{Pipes- und Filter- Struktur der Webapplikation}{pipesAndFilter}

\subsubsection{Segmentation-Filter}
Der Segmentation-Filter teilt die Tracksegmente der einzelnen Tracks in Teile, welche mit hoher Wahrscheinlichkeit nur mit einem Verkehrsmittel bewältigt wurden. Dazu verwendet dieser ermittelte Geschwindigkeits- sowie Beschleunigungswerte. Dieser Vorgang stützt sich auf die Ergebnisse aus den Publikationen von Zheng \cite{zheng_understanding_2010} und Biljecki \cite{biljecki_transportation_2013}. Der genaue Ablauf dieses Prozesses ist im Abschnitt ``\ref{segmentierung} \nameref{segmentierung}`` beschrieben.

\subsubsection{TravelMode-Filter}
In diesem Filter wird der eigentliche Verkehrsmitteltyp anhand der berechneten Werte bestimmt. Dies geschieht in beiden Fällen (mit und ohne GIS-Daten) mit Hilfe des Entscheidungsbaumes (siehe Abschnitt ``\ref{entscheidungsbaum} \nameref{entscheidungsbaum}`` und  ``\ref{entscheidungsbaumGIS} \nameref{entscheidungsbaumGIS}``), welcher mit Hilfe der jeweiligen Trainingsdaten generiert worden ist. Der genaue Ablauf dieses Prozesses ist im Abschnitt ``\ref{entscheidungsbaum_verwendung} \nameref{entscheidungsbaum_verwendung}`` beschrieben.

\subsubsection{Postprocessing-Filter}
Bei diesem Filter geht es darum, dass nach der Bestimmung aller Verkehrsmittel der Segmente eines Tracks, die Plausibilität und Sinnhaftigkeit der Wechsel nochmals überprüft wird. Diese letzte Überprüfung der Resultate der Entscheidungsbäume basiert auf Zheng \cite{zheng_understanding_2010} und soll verhindern, dass sehr unwahrscheinliche Verkehrsmittelwechsels wie zum Beispiel ``Auto->Bus->Auto->Bus->Auto`` entstehen. Basierend auf der Aussage von Zheng \cite{zheng_understanding_2010}, dass sich zwischen allen Verkehrsmittelwechsel ein (kleines) Segment vom Typ ``zu Fuß`` befinden muss, können Wechsel wie im obigen Beispiel verhindert werden. Somit wird das Resultat von ``Auto->Bus->Auto->Bus->Auto`` zu ``Auto->Auto->Auto->Auto->Auto`` korrigiert. Der genaue Ablauf dieses Prozesses ist im Abschnitt ``\ref{nachbearbeitung} \nameref{nachbearbeitung}`` beschrieben.

\subsubsection{DatabaseWriter}
Da das Ablegen der ermittelten Ergebnisse in der Datenbank  nur eine Variante (neben dem Ablegen  in einer Datei usw.) von vielen ist, kümmert sich diese Komponente darum, dass die übergebenen Daten für die Datenbank aufbereitet und schlussendlich persistiert werden.

\section{Verwendung der Applikation}
Die Oberfläche der Webapplikation ist sehr einfach gehalten und besteht im Wesentlich aus drei verschiedenen Seiten. Einer Startseite (Home), welche das Projekt kurz vorstellt, einer Seite (Create), auf welcher ein GPS-Track analysiert werden kann, und einer Seite (Results), auf welcher die Resultate aller Auswertungen angezeigt werden.

\subsection{Create-Seite}
Auf dieser Seite wird dem Benutzer/der Benutzerin die Möglichkeit geboten, einen Track hochzuladen und diesen mit einem von den zwei Varianten (mit und ohne GIS-Daten) analysieren zu lassen. Ist dies geschehen, so wird eine Karte angezeigt, auf welcher die ausgelesenen Segmente dargestellt werden. Diese Segmente werden, je nach Verkehrsmitteltyp, in verschiedenen Farben visualisiert. Per Klick auf eines dieser Segmente kann der Verkehrsmitteltyp des Segments korrigiert werden (siehe Abbildung \imgref{webapp1}). Diese Korrektur hat wiederum Einfluss auf die Auswertung der Ergebnisse, welche in Abbildung \imgref{webapp2} ersichtlich sind.

\img{0.75}{document/graphics/webapp1.png}{Korrigieren eines Verkehrsmitteltyps eines analysierten Tracks}{webapp1}

\subsection{Results-Seite}
\label{results-seite}
Die Resultate aller vorgenommenen Analysen und Änderungen werden auf dieser Seite als Diagramme dargestellt. Dabei zeigt das erste Diagramm das Verhältnis aller analysierten zu den korrekt erkannten Segmenten. Das zweite Diagramm zeigt die Anzahl der korrekt analysierten Segmente und die Gesamtanzahl der Segmente je nach Verkehrsmittel. Alle weiteren Diagramme betreffen ein spezifisches Verkehrsmittel und geben darüber Auskunft, wie oft das jeweilige Verkehrsmittel richtig bzw. nicht richtig erkannt worden ist. Dabei werden die nicht richtig erkannten Resultate nach Verkehrsmitteln sortiert.

Alle Diagramme zeigen dabei die Werte für sämtliche verfügbaren Analysemethoden an, um einen direkten Vergleich zu ermöglichen. Ein Ausschnitt dieser Seite ist in Abbildung \imgref{webapp2} zu sehen. In dieser Abbildung sind auch die Resultate der Untersuchungen mit verschiedenen Kombinationen der Zusatzvariablen sichtbar (siehe Abschnitt \ref{auswertung} \nameref{auswertung}).

\img{0.75}{document/graphics/webapp2.png}{Ausschnitt zu den Resultaten der analysierten Tracks}{webapp2}

\section{Konfiguration des Prototyps}
Der Prototyp lässt sich in den verschiedenen Bereichen weitestgehend konfigurieren. Diese Konfiguration kann in einer separaten Datei (app/config/config.yml) vorgenommen werden. Die Konfigurationsmöglichkeiten beinhalten unter anderem:
\begin{pitemize}
\item Konfiguration des Default-Namespaces für das Parsen der GPX-Dateien, wenn keiner in der Datei gefunden wurde.
\item Verschiedenste Grenzwerte zum Beeinflussen des Filterns von fehlerhaften Ausreißern in den Trackpunkten.
\item Diverse Grenzwerte für den Segmentierungsprozess, welcher die einzelnen Tracks in Segmente mit nur einem Verkehrsmittel unterteilt.
\item Definitionsmöglichkeiten für alle Analysemethoden sowie deren Eigenschaften.
\end{pitemize}

\subsection{Konfiguration des Filters für fehlerhafte Ausreißer}
In diesem Konfigurationsbereich (siehe Listing \coderef{filterConfig}) inkludiert sind sowohl eine minimale als auch eine maximale Distanz in Metern pro Zeiteinheit sowie ein Maximalwert für die Änderung der Höhenmeter. Weiters kann hier definiert werden, wie hoch die minimale Anzahl der Trackpunkte pro Segment sein muss, wie groß die minimale Zeitdifferenz (in Sekunden) zwischen zwei Trackpunkten sein soll und wie viele Punkte am Start übersprungen werden sollen. Außerdem kann ein Prozentwert für die Anzahl der validen Trackpunkte im Verhältnis zu allen Trackpunkten eingestellt werden. Eine genauere Erklärung zum Filterprozess ist im Anhang 1 zu finden.

\begin{code}[]{Filterkonfiguration}{filterConfig}
  filter:
    min_distance: 0.1
    max_distance: 50
    max_altitude_change: 25
    min_trackpoints_per_segment: 2
    min_time_difference: 2
    points_to_skip_from_start: 2
    min_valid_points_ratio: 0
\end{code}

\subsection{Konfiguration des Segmentierens}
Im Abschnitt der Segmentierungskonfiguration (siehe Listing \coderef{segmentConfig}) kann festgelegt werden, bis zu welcher Geschwindigkeit (m/s) und Beschleunigung (m/s\textsuperscript{2}) ein Wegpunkt als Geh-Punkt gilt. Weiters kann festgelegt werden, welches die minimale Zeitspanne (in Sekunden) und die minimale Distanz (in Metern) ist, die ein Segment haben muss, um nicht mit dem vorangegangenen Segment vereint zu werden. Schlussendlich gibt es drei Werte, welche zum Beenden eines Segments führen können. Darunter ist sowohl ein Stopp, welcher durch eine sehr geringe oder gar keine Bewegung über eine bestimmte Zeitspanne (in Sekunden) definiert wird als auch eine Zeitspanne (in Sekunden), in welcher keine neuen Trackpunkte gefunden werden. Wofür diese Werte benötigt werden, wird im Abschnitt ``\ref{segmentierung} \nameref{segmentierung}`` detailliert erklärt. Die letzten zwei Konfigurationsvariablen beschreiben die minimale Zeit und Distanz, die ein Segment haben muss, um vom Status eines ungewissen Segments in den Status eines sicheren übergehen zu können.

\begin{code}[]{Segmentierungskonfiguration}{segmentConfig}
  segmentation:   
    max_walk_velocity: 2.78
    max_walk_acceleration: 1.5
    
    min_segment_time: 20
    min_segment_distance: 50
    max_time_difference: 30
    
    max_time_without_movement: 10
    max_velocity_for_nearly_stoppoints: 0.55
    
    certain_segments_min_time: 60
    certain_segments_min_distance: 100
\end{code}

\subsection{Konfiguration der Analysemethoden}
Für die Analyse können verschiedenste Methoden definiert werden. In Listing \coderef{analyseConfig} ist die Konfiguration für die Analysemethoden ohne GIS-Daten abgebildet. Hierbei werden unter einem Namen für die Analysemethode (basic) verschiedene Konfigurationsvariablen abgelegt, darunter zum einen Variablen für die Erstellung des Entscheidungsbaums, zu anderen Variablen für die Generierung der Datei mit den Trainingsdaten. Die Konfiguration für den Entscheidungsbaum umfasst einen Klassennamen (zugleich auch Name der generierten Datei), das Verzeichnis, in welchem die generierte Entscheidungsbaum-Datei abgelegt werden soll, sowie den Pfad, wo die Textdatei mit der Definition des Entscheidungsbaums gefunden wird. Wozu diese Werte benötigt werden, wird im Abschnitt ``\ref{entscheidungsbaumGenerierungPHP} \nameref{entscheidungsbaumGenerierungPHP}`` genauer erklärt. Hinter der Konfigurationsvariable ``csv\_columns`` verstecken sich die Spalten, welche beim Erstellen der Trainingsdatendatei verwendet werden.  Diese definieren somit auch, welche Entscheidungsvariablen im Entscheidungsbaum verwendet werden.

\begin{code}[]{Analysekonfiguration}{analyseConfig}
    basic:
      class: "BasicDecisionTree"
      cacheDir: "%kernel.root_dir%/../decisionTrees/basic"
      txtFilePath: "%kernel.root_dir%/../decisionTrees/basic"
      txtFileName: "basicDecisionTree.txt"
      csv_columns:
        - "stoprate"
        - "meanvelocity"
        - "meanacceleration"
        - "maxvelocity"
        - "maxacceleration"
\end{code}
