\chapter{Einleitung}
Eine der mittlerweile wertvollsten Ressourcen die der Mensch selbst produziert, sind Daten. Der Begriff Daten umfasst in diesem Kontext alle Informationen zur eigenen Person wie zum Beispiel Name und Adresse aber auch die eigenen Interessen, die persönliche politische Einstellungen, Informationen darüber wann man sich wo aufgehalten hat und viel mehr. Auch in der breiten Bevölkerung bekommt diese Tatsache immer mehr Wichtigkeit beigemessen wie man an Spielen wie dem \todo{datadealer zitat}Data-Dealer erkennen kann.

Viele dieser Daten werden sowohl  bewusst als auch unbewusst mit Anderen geteilt oder auf eine andere Art und Weise in Umlauf gebracht. Das Stichwort Datenkrake ist allgegenwärtig. Ein Beispiel dafür, dass sich mit solchen Daten aber auch Ziele verfolgen lassen, ohne einzelne Personen speziell zu analysieren, soll diese Arbeit über das automatisierte Erkennen von verschiedenen Verkehrsmitteln sein. 

\section{Ziele dieser Arbeit}
Das Hauptziel dieser Arbeit ist es, einen Prototypen zu erstellen, welcher anhand von aufgezeichneten GPS-Spuren und in Kombination mit verschieden Methodiken versucht das benutzte  Verkehrsmittel mit einer möglichst hohen Wahrscheinlichkeit zu bestimmen. Diese aufgezeichneten GPS-Spuren enthalten dabei keinerlei Informationen über die jeweilige Person. Deshalb erfolgt die Auswertung ausschließlich über die GPS-Spur sowie über öffentliche zugängliche Daten (z.B. Busstationen und Gleise). Jede dieser Aufzeichnung kann mehrere Verkehrsmittel beinhalten und von unterschiedlicher Länge sein.

Es soll weiters, für jede Person die ein Smartphone besitzt möglich sein, selbst solche GPS-Spuren aufzuzeichnen und diese analysieren zu lassen. Dies bedeutet, dass keine speziellen Geräte oder andere Sensoren benötigt werden und sich dieser Prototyp mit möglichst geringen Anpassungen auch auf andere Regionen (mit technisch weniger ausgereiften Smartphones) anwenden lassen kann. Zum Aufzeichnen der GPS-Spuren wurde für diese Arbeit die App MyTrack verwendet aber im wesentlichen funktioniert der Prototyp mit allen GPX-konformen Dateien unabhängig davon von welchem Gerät diese aufgezeichnet worden sind.

Im Zuge dieser Arbeit wird auch untersucht welches Level an Genauigkeit sowohl mit als auch ohne geografischen Informationen im Raum Vorarlberg erreicht werden kann. Dabei soll es nach der Analyse die Möglichkeit geben, die automatisch bestimmten Verkehrsmittel manuell zu korrigieren sollten diese nicht mit der Realität übereinstimmen. Auch diese manuellen Änderungen sollen in die Auswertung mit einfließen. 

Schlussendlich soll auch eine Aussage über die Machbarkeit beziehungsweise den Aufwand und die möglicherweise noch zusätzlich benötigten Daten gemacht werden.

\subsubsection{Nichtziele}

In dieser Arbeit werden die diverse Schlussfolgerungsmodelle nicht betrachtet oder verglichen sondern es wird auf ein aufgrund von anderen Arbeiten vielversprechendes Modell gesetzt. Dieses Modell ist der Entscheidungsbaum. Außerdem werden die Daten zur Analyse bzw. Auswertung nicht in Echtzeit betrachtet sondern in Form einer GPS-Spur (Log) an den Prototypen übergeben.

Die Frage, wo am Körper man das Smartphone am besten trägt um genaue GPS-Daten zu erhalten sowie die Frage nach dem Energierverbrauch der App bzw. wie eine möglichst energieschonende App und die dazugehörige Kommunikation aufgebaut sein könnte bleibt unberücksichtigt. Auch eine umfassende Behandlung der Themen Sicherheit und Privatspähre würden den Rahmen der vorliegenden Arbeit sprengen und bleibt daher unberührt.

\section{Motivation und Nutzen}
Schon früher wurde versucht Aufzeichnungen über die Verkehrswege von verschiedenen Menschen zu sammeln. Aber die Protokollen in Papierform sowie die Telefonbefragungen waren zu aufwändig und die Probanden nicht zuverlässig genug. Darum ist es von entscheidendem Vorteil eine App zur Verfügung zu haben, welche die Vorgänge des Aufzeichnens möglichst genau für einen übernimmt.

Wird eine Auswertung mit einer für das Zielgebiet aussagekräftigen Anzahl an Personen durchgeführt, so kann das Resultat  für verschiedenste Zwecke verwendet werden. Auch ohne  Analyse kann rein durch die Betrachtung der gesammelten GPS-Spuren festgestellt werden, welche Routen besonders häufig benutzt werden. 

Zieht man nun verschiedene Werte aus der Auswertung hinzu kann auch festgestellt werden wo sich zum Beispiel verkehrstechnische Engstellen befinden und welche Routen sehr populär sind oder Aussagen über die allgemeine Verkehrssituation machen. Durch die gesammelten Daten könnten sich auch Simulationen für anstehende Bauvorhaben machen lassen und auch versucht werden eine Vorhersage für bestimmte Situationen zu tätigen. Diese Aspekte können unter anderem für das Verkehrsministerium,  den öffentlichen Personennahverkehr oder auch für die Stadtplanung (\todo{Zitat kairo} siehe Kairo) sehr interessant sein.

Eine ganze Reihe von Apps lässt sich mit den Auswertungen erstellen. Diese Apps könnten die Auswertungen in soziale Medien zu integrieren, für Fitnessanalysen verwendet werden, einen einfachen Rückblick über die eigene Fortbewegung ermöglichen oder für Umweltbewusste errechnen wie viel CO2 sie produziert oder gespart haben. Ein Reisetagebuch könnte daraus genau so Nutzen ziehen wie eine App die beim Autofahren Auskunft über die aktuell billigste Tankstelle in näherer Umgebung gibt oder eine App die einfach nur Vorschläge für alternative Routen zu einem bekannten Ziel anbietet.

Zusammenfassend kann man sagen, dass die Verwendungsmöglichkeiten für solche Daten umfangreich sind und sich am besten unter den Begriffen kontextorientierte, geographische Apps zusammenfassen lassen. Nicht zuletzt öffnen sich mit solchen Daten aber auch umfangreiche Möglichkeiten für die Werbebranche.

\section{State of the art}
Zu den Meilensteinen auf diesem Gebiet zählet sicher die Arbeit von \todo{zheng zitat}Zheng in welcher er unter anderem auf die Erkennung von den Abschnitten mit nur einem Verkehrsmittel eingeht. Weiters verwendete er in seiner Arbeit auch einen mit dem von den GPS-Spuren gesammelten geographischen Wissen aufgebauten Graphen welcher zur weiteren Auswertung verwendet wurde.

Mit der Frage wie geographische Daten in eine solche Analyse miteinbezogen werden können hat sich auch  \todo{Stenneth zitat}Stenneth beschäftigt. Dabei hat er nicht nur fixe Daten wie Gleise und Busstationen sondern auch aktuelle Buspositionen miteinbezogen. 

Sowohl Stenneth als auch Zheng haben in ihren Arbeiten detailiert erklärt wieso sie welche Attribute (Geschwindigkeit, Beschleunigung, ...) für die Bestimmung des Verkehrsmittels verwendet haben und sie haben diese auch durch Versuche nach ihrer Wichtigkeit gereiht. Außerdem haben beide und auch Reddy \todo{reddy zitat} mehrere Schlussfolgerungsmodelle (Entscheidungsbaum, Bayessches Netz, Markov Modelle, Random Forest, ...) betrachtet und miteinander verglichen.

Wie man mit Verbindungsabbrüchen umgeht und zwischen ähnlichen Verkehrsmitteln unterscheiden kann hat unter anderem auch \todo{Biljeki zitat}Biljecki beschäftigt. Außerdem baut er für die unterschiedlichen Kategorien von Verkehrsmittel ein hierarchisches Modell auf welches ihm helfen soll bessere Entscheidungen zu treffen.

\subsection{Daten}
Ein wesentlicher Unterschied zwischen all den betrachteten Publikationen sind die verwendeten Daten. Einzig die GPS-Spuren bilden eine gemeinsame Basis. Manche untersuchten die Verwendung von \todo{gsm und wifi daten zitat} GSM- und WIFI-Informationen,  stützten sich auf Zusatzinformationen durch weitere Sensoren wie zum Beispiel ein \todo{beschleunigungssensoren zitat}Beschleunigungssensor. Andere wie \todo{stenneth zitat}Stenneth verwendeten Live-Informationen von den öffentlichen Verkehrsmittel und kombinierten diese mit GIS-Informationen, seien es Busstationen, Bahnstrecken, das Straßennetz oder Parkplätze.

\subsection{Verkehrsmittel}
Ein weiterer Unterschied zwischen den Publikationen sind die betrachteten Verkehrsmittel. Hierbei reicht die Spanne der unterschiedenen Verkehrsmittel von "Gehen und Motorisiert" \todo{reddy zitat}siehe Reddy bis hin zu "Gehen, Zug, U-Bahn, Rad, Auto, Straßenbahn, Bus, Fähre, Segelbot und Flugzeug" \todo{biljecki zitat} siehe Biljecki.

\section{Weiterer Aufbau der Arbeit}
Der Hauptteil der vorliegenden Arbeit gliedert sich in 4 große Abschnitte:

Im ersten Abschnitt wird eine Terminiologie für die weiteren Abschnitte festgelegt um Missverständnisse und Mehrdeutigkeiten zu beseitigen. Danach wird auf die Akquirierung der GPS-Daten eingegangen. Dies umfasst sowohl die gesammelt GPS-Spuren und deren Struktur, sowie die verwendeten GIS-Daten. Dabei geht es einerseits um deren Herkunft als auch darum wie diese extrahiert wurden und wie auch diese Daten aufgebaut sind. Schlussendlich wird ein Überblick über den entstandenen Prototypen sowie dessen Handhabung und Architektur vermittelt. 

Der zweite Abschnitt befasst sich sowohl mit dem Filtern als auch dem Aufbereiten der Daten sowie das Aufteilen von GPS-Spuren in Teile in denen nur ein Verkehrsmittel verwendet wurde. Filtern der Daten bedeutet in diesem Zusammenhang, dass Außreiser aus den GSP-Spuren entfernt werden. Diese Ausreißer können aufgrund von Geschwindigkeitssprüngen als auch unwahrscheinlich große Sprünge im dreidimensionalen Raum sein. Mit Aufbereiten der Daten ist gemeint, dass Werte zu GPS-Punkten für die spätere Analyse berechnet werden. Diese Werte können sowohl die Geschwindigkeit, Beschleunigung, Distanz, Zeit als auch der Abstand zu der nächsten Bushaltestelle sein. Schlussendlich sollen die gefilterten und erweiterten Daten verwendet werden um eine einfache und sichere Aufteilung der GPS-Spuren anhand der Verkehrsmittel zu ermöglichen.

Aufbauen auf die Resultate aus dem zweiten Abschnitt befasst sich der dritte Abschnitt mit der tatsächlichen Erkennung von der Verkehrsmittel anhand der berechneten Werte und den einzelnen Abschnitten der GPS-Spur. Dabei wird auch genauer auf die verwendete Schlussfolgerungsmethode den Entscheidungsbaum sowie dessen Training eingegangen. Die aus dem Entscheidungsbaum gewonnenen Erkenntnisse werden schlussendlich ein letztes Mal überprüft um z.B. sinnfreie Wechsel zwischen Verkehrsmitteln zu verhindern.

Im vierten und letzten Abschnitt des Hauptteils wird die Auswertung der gewonnenen Erkenntnisse vorgenommen. \todo{Auswertung Überblick}
