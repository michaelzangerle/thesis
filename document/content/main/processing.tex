\chapter{Filter}
Wie auch bei vielen anderen Arbeiten konnte auch bei dieser Arbeit festgestellt werden, dass sich in den GPS-Spuren einige Ausreißer befanden. Dies konnte vor allem dann beobachtet werden, wenn man sich in einem Zug befand, durch einen Tunnel fuhr oder auch wenn man sich auf einem überdachten Bahnsteig befand. Die Ausreißer werden durch unrealistisch große Distanzabstände, zu kleine Zeitabstände oder Sprünge in den Höhenwerten bemerkt. Da diese Werte das Ergebnis verfälschen würden, mussten verschiedene Filter implementiert werden. 

Konkret wurden Filter für Zeit, Distanz und Höhenmeter implementiert. Die bereinigten Resultate hatten bereits auf den Entscheidungsbaum große Auswirkungen. Die Grenzwerte sind in der Konfiguration festgelegt und können je nach Region und Testdaten angepasst werden.

Punkte werden insofern gefiltert, als dass sie übersprungen werden und der aktuelle Punkt mit dem nachfolgenden des Ausreißers verglichen wird. Die übersprungenen Punkte werden dann nicht für die weitere Verarbeitung berücksichtigt. Abgesehen von Zeitfilter betrachten alle anderen Filter die gemessen Werten in Relation zur gemessenen Zeit was bedeutet, dass der Zeitwert größer als 0 sein muss.

\section{Verschiedenen Fälle beim Filtern}
Es gibt drei verschiedene Fälle welche beim Filtern von Ausreißern abgedeckt werden sollten. Der grundlegende Algorithmus welcher vom aktuellen Punkt ausgehend einen neuen gültigen Punkt sucht und alle ungültigen überspringt funktioniert in den ersten zwei Fällen. Im dritten Fall muss noch eine zusätzliche Überprüfung stattfinden.

\textbf{1. Fall}\\
Beim ersten Fall befindet sich ein oder mehrere Ausreißer am Ende der GPS-Spur wie es in Abbildung \imgref{filterCase1} bei dem letzten Punkt der Fall ist. Dies bedeutet, dass ab einem gewissen Punkt keine weiteren validen Punkte gefunden werden und alle folgende Punkte übersprungen werden.

\img{0.75}{document/graphics/filterCase1.png}{Filtern - 1. Fall}{filterCase1}

\textbf{2. Fall}\\
Beim zweiten Fall befinden sich ein oder mehrere Ausreißer zwischen validen vorangegangenen und nachfolgenden Punkten. Ein Beispiel ist in Abbildung \imgref{filterCase2} mit dem vierten Punkt aus Ausreißer abgebildet. Dies bedeutet, dass ein oder mehrere Punkte übersprungen werden und danach mit den gültigen Punkten weitergearbeitet werden kann.

\img{0.75}{document/graphics/filterCase2.png}{Filtern - 2. Fall}{filterCase2}

\textbf{3. Fall}\\
Der dritte Fall beschäftigt sich mit ein oder mehreren Ausreißern am Beginn der GPS-Spur. Damit ist gemeint, dass vom Start weg keine gültigen Punkte vorhanden sind und erst im Laufe der Aufzeichnung gültige Punkte aufgezeichnet werden. Dies kann vorkommen, wenn die Aufzeichnung der GPS-Spur sofort nach dem Aktivieren des GPS-Moduls startet. Die Position konnte noch nicht mit ausreichender Genauigkeit bestimmt werden und es wird mit einer niederen Genauigkeit gestartet. Im Laufe der Aufzeichnung steigt die Genauigkeit und es kann zu einem Sprung von ungenauen zu den genauen Punkten kommen. Ein Beispiel hierfür ist in Abbildung \imgref{filterCase3} mit dem ersten Punkt als Ausreißer ersichtlich.

\img{0.75}{document/graphics/filterCase3.png}{Filtern - 3. Fall}{filterCase3}

\section{Zeitfilter}
Der Zeitfilter überprüft ob der Abstand zwischen zwei GPS-Punkten größer gleich einem minimalen Wert ist. Dadurch wird verhindert, dass zwei Punkte mit demselben Zeitstempeln verarbeitet werden und bei den zeitabhängigen Berechnungen durch 0 dividiert wird. Außerdem kann man dadurch auch steuern wie viele Punkte pro GPS-Spur überprüft werden  (z.B. nur jeder 2. Punkt) beziehungsweise welche Punkte ausgelassen werden sollen um den Prozess zu beschleunigen oder weil sich der Grad an Genauigkeit nicht wesentlich verbessert.

\section{Distanzfilter}
Der Distanzfilter kontrolliert ob sich der Abstand zwischen zwei Punkten im Verhältnis zur Zeit in einem gewissen Bereich befindet. In dieser Arbeit wurde größer 0 m/s als minimale und kleiner 50 m/s als maximale Distanz festgelegt. Liegt ein Punkt nicht innerhalb dieser Grenzen so wird der aktuelle Punkt mit den Punkt nach dem Ausreißer verglichen. Dies wird solange gemacht bis wieder ein Punkt mit valider Distanz gefunden wird oder keine GPS-Punkte mehr vorhanden sind.

\section{Höhenfilter}
Der Höhenfilter filtert ähnlich wie der Geschwindigkeitsfilter jene GPS-Punkte, bei welcher Differenz der Höhenwerte zu groß ist. In Fall der hier verwendeten Trainingsdaten wurde 25 m/s  für diesen Filter festgelegt und alle Punkte mit einen größeren Differenz werden herausgefiltert.

\pagebreak

\section{Aufbereitung ohne GIS-Daten}
\subsection{Auswahl}
\subsubsection{Durchschnittliche Geschwindigkeit}
\subsubsection{Maximale Geschwindigkeit}
\subsubsection{Durchschnittliche Beschleunigung}
\subsubsection{Maximale Beschleunigung}
\subsubsection{Höhenmeter}
\subsubsection{Distanz}
\pagebreak

\section{Aufbereitung mit GIS-Daten}
\subsection{Auswahl}
\subsubsection{Durchschnittliche Geschwindigkeit}
\subsubsection{Maximale Geschwindigkeit}
\subsubsection{Durchschnittliche Beschleunigung}
\subsubsection{Maximale Beschleunigung}
\subsection{Abstand zu Bushaltestellen}
\subsection{Abstand zu Gleisen}
\pagebreak

\section{Segmentierung}
\subsection{Terminologie}
