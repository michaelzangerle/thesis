\chapter{Auswertung}
\label{auswertung}

Zur Auswertung der Ergebnisse der zwei Analysemethoden wurden, wie bereits in Abschnitt ``\ref{results-seite}  \nameref{results-seite}`` erwähnt, mehrere Schnittstellen und schlussendlich auch Diagramme erstellt. Diese geben einen grundlegenden Überblick über die Resultate der verschiedenen Analysemethoden. Dabei wird zwischen den Gesamtwerten aller Verkehrsmittel sowie den einzelnen Verkehrsmitteln unterschieden. Dadurch lassen sich Tendenzen bezüglich der falschen Klassifizierungen erkennen. 
 
Neben diesen zwei Analysemethoden wurde auch mit der Auswahl der Zusatzwerte für die GIS-Analysemethode experimentiert. Damit wurde überprüft, welche Zusatzwerte welche Auswirkungen auf die Verkehrsmittelerkennung haben. 

Darüber hinaus wurde untersucht, welche Erkennungsraten mit Hilfe der unterschiedlichen Algorithmen zur Erstellung eines Entscheidungsbaums bzw. zur Findung des nächsten Attributs zur Erstellung eines Entscheidungsbaums, erreicht werden können. Dies wurde in erster Instanz in RapidMiner und in zweiter Instanz mit dem Prototyp analysiert.
\clearpage

\section{Erstellung des Entscheidungsbaums}
\label{rapidMinerResultat}
Bei der Erstellung eines Entscheidungsbaums bzw. zur Bestimmung des nächsten Attributs bei der Erstellung eine Entscheidungsbaums kann in RapidMiner auf verschiedene Algorithmen bzw. Berechnungsarten zurückgegriffen werden. Darunter befinden sich der Informationsgewinn, der Gini-Index und die Informationszugewinn-Rate. Es wurden in weiterer Folge verschiedene Entscheidungsbäume mit verschiedenen Varianten der Zusatzvariablen und mit unterschiedlichen Algorithmen generiert. Neben den Entscheidungsbäumen wurde mit Hilfe von RapidMiner auch berechnet, welche Erkennungsrate mit diesen Entscheidungsbäumen zu erwarten ist. Dabei stützt sich dieser Wert nur auf die verarbeiteten Trainingsdaten. Die Resultate sind in Tabelle \ref{dt-calculation-methods} zu sehen. 

\begin{table}[h]
\centering
\begin{tabular}{|l|c|c|c|}
\hline
 & {\bf Informationsgew.} & {\bf Gewinn-Rate} & {\bf Gini-Index} \\ \hline
Ohne GIS-Daten & 61,40 & 66,67 & 71,93 \\ \hline
Mit GIS-Daten & 64,91 & 64,91 & 73,68 \\ \hline
Mit GIS-Daten ohne max. Beschl. & 64,91 & 59,65 & 73,68 \\ \hline
Mit GIS-Daten ohne max. Geschw. & 64,91 & 64,91 & 73,68 \\ \hline
Mit GIS-Daten ohne max. Werte & 68,42 & 61,40 & 73,68 \\ \hline
Mit GIS-Daten ohne mittl. Beschl. & 71,93 & 64,91 & 71,93 \\ \hline
Mit GIS-Daten ohne mittl. Geschw. & 61,40 & 63,16 & 68,42 \\ \hline
Mit GIS-Daten ohne Mittelwerte & 61,40 & 57,89 & 57,89 \\ \hline
Mit GIS-Daten ohne Geschw. & 61,40 & 61,40 & 68,42 \\ \hline
Mit GIS-Daten ohne Beschl. & 70,18 & 59,65 & 71,93 \\ \hline
Mit GIS-Daten ohne Gleis & 64,91 & 63,16 & 68,42 \\ \hline
Mit GIS-Daten ohne Autobahn & 66,67 & 61,40 & 70,18 \\ \hline
Mit GIS-Daten ohne Bushaltest. & 64,91 & 61,40 & 70,18 \\ \hline
Mit GIS-Daten ohne Stoprate & 71,93 & 64,91 & 75,44 \\ \hline
\end{tabular}
\caption{Genauigkeit von Entscheidungsbäumen mit verschiedenen Algorithmen und Trainingsdatenvarianten}
\label{dt-calculation-methods}
\end{table}

Die Tabelle \ref{dt-calculation-methods} zeigt, dass sich die zu erwartende Erkennungsrate zwischen 61\% und fast 74\% befindet. Dabei wird mit dem Gini-Index das beste Ergebnis (70,68\% im Schnitt) und mit der Zugewinn-Rate (im Schnitt 62,53\%) das schlechteste erzielt. Der Informationsgewinn liegt mit 65,66\% dazwischen.

Weiters erkennt man, dass die verschiedenen Zusatzwerte unterschiedlich viel Einfluss auf die allgemeine Erkennung haben. So fällt die Erkennungsrate mit dem Gini-Index von 73,68\% auf 57,89\%, wenn man die berechneten Mittelwerte für Geschwindigkeit und Beschleunigung nicht miteinbezieht. Die maximalen Werte für Beschleunigung und Geschwindigkeit bewirken sowohl zusammen als auch getrennt wenig bis keinen Unterschied für die Erkennung,  wenn man den Gini-Index verwendet. Interessant ist hingegen, dass die GIS-Analysemethode ohne Stopprate mit dem Gini-Index auf 75,44\% kommt.

Bei den Werten für die Berechnung mittels Informationsgewinn lässt sich noch hervorheben, dass die Testdatenvarianten ohne Beschleunigungswerte bzw. ohne mittlere Beschleunigung Erkennungsraten über von 70\% erzielten. 

Erkennen lässt sich weiters, dass zwischen der Variante mit und der ohne GIS-Daten nur 2\% Prozent liegen und diese im Falle der Zugewinn-Rate sogar für die Variante ohne GIS-Daten sprechen. Diese Werte sind jedoch nur auf Basis der Testdaten errechnet und müssen nicht mit der tatsächlichen Erkennungsrate übereinstimmen. Deshalb werden im nächsten Abschnitt die Resultate des Prototyps genauer betrachtet.

\section{Analyse mit dem Prototyp}
Für die Analyse mit Hilfe des Prototyps wurden neben den zwei Analysemethoden mit und ohne GIS-Daten noch drei weitere Entscheidungsbäume ausgewählt. Die wurden auf Grund ihres Abschneidens in Abschnitt ``\ref{rapidMinerResultat} \nameref{rapidMinerResultat}`` ausgewählten Entscheidungsbäume und sind:

\begin{pitemize}
\item mit GIS-Daten ohne Maximalwerte
\item mit GIS-Daten ohne Mittelwerte
\item mit GIS-Daten ohne Gleise
\item mit GIS-Daten ohne Stopprate
\end{pitemize}

\subsection{Gesamtresultate}
Wie in Tabelle \ref{total-results} ersichtlich ist, schneiden alle Analysemethoden mit GIS-Daten besser ab als jene ohne. Mit der Analysemethode ohne GIS-Daten wurden  66,30\% richtig erkannt. Die Analyse mit GIS-Daten konnte im Gensatz dazu fast 69\% richtig erkennen. Die Auswirkungen der verschiedenen Variationen der Entscheidungsvariablen für die GIS-Analyse werden auch durch dieses Resultat sehr deutlich. So kann die Variante ohne Gleis-Daten 67,71\% richtig erkennen, aber die Varianten ohne Durchschnittswerte oder ohne Stopprate erkennen fast 75\% richtig.

Es wird deutlich, dass verschiedene Kombinationen von Entscheidungsvariablen sehr unterschiedliche Auswirkungen auf die tatsächliche Erkennung haben. Dies bestätigt auch die Aussagen von Zheng und Stenneth, wonach zu viele Variablen das Ergebnis verschlechtern können. \cite{zheng_understanding_2010} \cite{stenneth_transportation_2011}

\begin{table}[h]
\centering
\begin{tabular}{|l|c|c|}
\hline
 & \# & \% \\ \hline
Gesamt & 1.347 & 100,00 \\ \hline
Ohne GIS-Daten & 893 & 66,30 \\ \hline
Mit GIS-Daten & 926 & 68,75 \\ \hline
Mit GIS-Daten ohne Maximalwerte & 953 & 70,75 \\ \hline
Mit GIS-Daten ohne Durchschnittswerte & 1.004 & 74,54 \\ \hline
Mit GIS-Daten ohne Gleise & 912 & 67,71 \\ \hline
Mit GIS-Daten ohne Stopprate & 997 & 74,02 \\ \hline
\end{tabular}
\caption{Gesamtresultate im Überblick}
\label{total-results}
\end{table}

\subsection{Detailresultate}
Die Detailresultate sind spezifisch auf ein Verkehrsmittel ausgerichtet und betrachten die richtig klassifizierten Segmente dieses Verkehrsmittels. Ihnen gegenüber stehen die fälschlicherweise als der jeweilige Verkehrsmitteltyp klassifizierten Segmente. Diese werden auf die eigentlich richtigen Verkehrsmittel aufgeteilt.

\textbf{Bus} \\
Für die Segmente, die laut Prototyp mit einem Bus bewältigt worden sind, kann festgestellt werden, dass ein paar wenige Segmente eigentlich Fahrrad- oder Auto-Segmente gewesen sind. Weiters wurden ohne GIS-Daten, ohne Mittelwerte und ohne Stopprate deutlich weniger Bus-Segmente erkannt wie mit den anderen Entscheidungsbäumen. Ein Auszug der Statistik über die richtig erkannten Bus-Segmente ist in Tabelle \ref{result-bus} zu sehen und zeigt, dass Bus-Segmente mit einer Wahrscheinlichkeit von 36,19\% mit der GIS-Analyse erkannt werden.

\begin{table}[h]
\centering
\begin{tabular}{|l|c|}
\hline
 & \% \\ \hline
Ohne GIS-Daten & 21,90 \\ \hline
Mit GIS-Daten & 36,19 \\ \hline
Mit GIS-Daten ohne Maximalwerte & 38,10 \\ \hline
Mit GIS-Daten ohne Durchschnittswerte & 18,10 \\ \hline
Mit GIS-Daten ohne Gleise & 35,24 \\ \hline
Mit GIS-Daten ohne Stopprate & 14,29 \\ \hline
\end{tabular}
\caption{Richtig erkannte Bus-Segmente}
\label{result-bus}
\end{table}

\textbf{Auto} \\
Bei den Segmenten, welche als Auto-Segmente klassifiziert worden sind, kann festgestellt werden, dass die Erkennungsrate ungleich höher ist und zwischen 75 und 86\% liegt. Die meisten falsch als Auto klassifizierten Segmente sind dabei Bus- und Fahrrad-Segmente. Die Ähnlichkeiten zwischen Bus und Auto war bereits bekannt und nicht weiter überraschend. Die häufige Fehlklassifizierung von Fahrrad-Segmenten als Auto-Segmente führt jedoch zu der Erkenntnis, dass einige Fahrrad-Segmente in den Trainingsdaten sind, welche über eine hohe Geschwindigkeit verfügen und deshalb eine Fehlklassifizierung ermöglichen. Ein Auszug der Statistik über die richtig erkannten Auto-Segmente ist in Tabelle \ref{result-drive} zu sehen und zeigt, dass die Erkennung von Auto-Segmenten für die GIS-Analyse bei 82,22\% liegt. Die Analyse-Methoden ohne Beschleunigung liefern für diesen Fall die beste Erkennungsrate mit 86,12\%.

\begin{table}[h]
\centering
\begin{tabular}{|l|c|}
\hline
 & \% \\ \hline
Ohne GIS-Daten & 77,83 \\ \hline
Mit GIS-Daten & 82,22 \\ \hline
Mit GIS-Daten ohne Maximalwerte & 86,14 \\ \hline
Mit GIS-Daten ohne Durchschnittswerte & 83,14 \\ \hline
Mit GIS-Daten ohne Gleise & 78,75 \\ \hline
Mit GIS-Daten ohne Stopprate & 75,52 \\ \hline
\end{tabular}
\caption{Richtig erkannte Auto-Segmente}
\label{result-drive}
\end{table}

\textbf{Fahrrad} \\
Die Resultate für die Fahrrad-Segmente zeigen, dass sich die Fehlklassifizierungen großteils in den Segmenten vom Typ ``zu Fuß`` und Auto befinden. Der Grund für die falschen Klassifizierungen als Auto-Segment ist derselbe wie bei den Resultaten zum Typ Auto. Schnelle Radsegmente, die z.B. durch einen Rennradfahrer entstanden sind, können nicht gut von langsameren Autofahrten z.B. in der Stadt unterschieden werden. Weiters ist es auch schwierig, eine sehr langsame Radfahrt z.B. wenn es bergauf geht von einem Fußgänger zu unterscheiden, da Geschwindigkeit, Beschleunigung und Stopprate von beiden Fortbewegungsarten sehr ähnlich sein können. Ein Auszug der Statistik über die richtig erkannten Fahrrad-Segmente ist in Tabelle \ref{result-bike} zu sehen und zeigt, dass 46,68\% der Fahrradsegmente mit Hilfe der GIS-Analyse erkannt werden. Alle anderen Analysen liefern in diesem Fall bessere Ergebnisse. Besonders die Variante ohne Stopprate liefert in diesem Fall ein wesentlich besseres Ergebnis mit 70\%.

\begin{table}[h]
\centering
\begin{tabular}{|l|c|}
\hline
 & \% \\ \hline
Ohne GIS-Daten & 52,49 \\ \hline
Mit GIS-Daten & 46,68 \\ \hline
Mit GIS-Daten ohne Maximalwerte & 47,72 \\ \hline
Mit GIS-Daten ohne Durchschnittswerte & 64,52 \\ \hline
Mit GIS-Daten ohne Gleise & 52,70 \\ \hline
Mit GIS-Daten ohne Stopprate & 70,33 \\ \hline
\end{tabular}
\caption{Richtig erkannte Fahrrad-Segmente}
\label{result-bike}
\end{table}

\textbf{Zu Fuß} \\
Die Ergebnisse für die ``zu Fuß``-Segmente sind sehr eindeutig und liegen für alle Analyse-Methode über 93\%.  Aber auch hier liefert die GIS-Analyse-Methode mit weniger Zusatzvariablen bessere Ergebnisse. Die falschen Klassifizierungen liegen hauptsächlich im Bereich der Fahrrad-Segmente und Auto-Segmente. Dies lässt in beiden Fällen auf eine sehr langsame Fortbewegungsgeschwindigkeit schließen. Bei den Fahrrad-Segmenten konnte dies nach stichprobenartigen Untersuchungen auch belegt werden, denn dabei handelt es sich um Mountainbike-Touren, die bergaufwärts naturgemäß langsamere Geschwindigkeiten haben. Ein Auszug der Statistik über die richtig erkannten ``zu Fuß``-Segmente ist in Tabelle \ref{result-foot} zu sehen.

\begin{table}[h]
\centering
\begin{tabular}{|l|c|}
\hline
 & \% \\ \hline
Ohne GIS-Daten & 93,93 \\ \hline
Mit GIS-Daten & 93,93 \\ \hline
Mit GIS-Daten ohne Maximalwerte & 94,64 \\ \hline
Mit GIS-Daten ohne Durchschnittswerte & 96,43 \\ \hline
Mit GIS-Daten ohne Gleise & 93,93 \\ \hline
Mit GIS-Daten ohne Stopprate & 97,14 \\ \hline
\end{tabular}
\caption{Richtig erkannte ``zu Fuß``-Segmente}
\label{result-foot}
\end{table}

\textbf{Zug} \\
Bei den Resultaten der Zug-Segmente konnten die GIS-Daten ihre Stärke voll ausspielen und im Gegensatz zu den Analyse-Methoden ohne GIS-Informationen von Gleisen, die eine ca. 35\%ige Erkennungsrate haben. Mit Hilfe der GIS-Daten über Gleise konnte die Erkennung auf bis zu 93\% gesteigert werden. Außerdem konnten auch die Fehlklassifizierungen als Auto-Segmente eliminiert werden. Ein Auszug der Statistik über die richtig erkannten Zug-Segmente ist in Tabelle \ref{result-train} zu sehen.

\begin{table}[h]
\centering
\begin{tabular}{|l|c|}
\hline
 & \% \\ \hline
Ohne GIS-Daten & 35,42 \\ \hline
Mit GIS-Daten & 91,67 \\ \hline
Mit GIS-Daten ohne Maximalwerte & 93,75 \\ \hline
Mit GIS-Daten ohne Durchschnittswerte & 91,67 \\ \hline
Mit GIS-Daten ohne Gleise & 35,42 \\ \hline
Mit GIS-Daten ohne Stopprate & 91,67 \\ \hline
\end{tabular}
\caption{Richtig erkannte Zug-Segmente}
\label{result-train}
\end{table}

\section{Zusammenfassung}
Wird das Gesamtergebnis der GIS-Analysevarianten betrachtet, so kann festgestellt werden, dass für Segmente vom Typ Auto, Zug und ``zu Fuß``, je nach Kombination der verwendeten Zusatzwerte,  gute Erkennungsraten (75-86\%, 91-93\%, 93-97\%) gefunden worden sind. Diese Verkehrsmittel können also  gut erkannt werden. Wo es Verbesserungsbedarf gibt, sind eindeutig die Bus- und Fahrrad-Segmente.  

Die Fahrrad-Segmente haben zum Teil schlecht abgeschnitten, weil die Trainings- und Testdaten für die Fahrrad-Segmente großteils von Sportlern stammen. Zum einen ist damit gemeint, dass im Flachen (Rennrad) und abwärts sehr hohe Geschwindigkeiten gemessen worden sind. Dies erklärt die Fehlklassifizierungen der Fahrrad-Segmente als Auto-Segmente. Zum anderen wurden bei Fahrrad-Segmenten auch sehr langsame Geschwindigkeiten gemessen. Dies ist dann vorgekommen, wenn es bergaufwärts ging. Dabei waren diese Abschnitte aufwärts meistens nicht kurz (so wie in einer Stadt) und hatten Charakteristika, die eher zu Geh-Segmenten passten als zu einem Fahrrad-Segment.

Weiters konnte beobachtet und schlussendlich auch bestätigt werden, dass zu viele Zusatzvariablen das Ergebnis verschlechtern können und mit weniger Variablen unter Umständen eine bessere Erkennungsrate erreicht werden kann. Dies bekräftigt auch die Aussagen von Zheng \cite{zheng_understanding_2010} und Stenneth \cite{stenneth_transportation_2011}, welche ihre verwendeten Zusatzvariablen evaluiert haben. Daraus ist bei beiden hervorgegangen, dass ab einer bestimmten Anzahl von Zusatzvariablen keine Verbesserung, sondern eher eine Verschlechterung des Ergebnisses zu erwarten ist. Dies lässt sich auch in den vorherigen Resultaten beobachten. Die fehlende Stopprate bei der Erkennung von Fahrradsegmenten bewirkt, dass deutlich mehr Fahrradsegmente richtig klassifiziert werden können. Im Gegensatz dazu steht allerdings das Resultat für die Analysevariante ohne Stopprate bei Bus-Segmenten. 

Bei der Betrachtung der Ergebnisse muss schlussendlich noch hervorgehoben werden, dass es sich bei den verwendeten Trainings- und Testdaten um überschaubare Datenmengen handelt und sich mit einer größeren Datenmenge wahrscheinlich bessere Erkennungsraten erzielen lassen werden. Dies kann damit begründet werden, dass bei kleinen Datenmengen wenige spezielle Testdaten, wie z.B. sehr langsame Fahrradsegmente (aufwärts) oder sehr schnelle Fahrradsegmente (abwärts, Rennradfahrer), wesentlich größeren Einfluss auf den Entscheidungsbaum und im Endeffekt auf die Erkennung haben. 
