\chapter{State of the Art}\todo{extend state of the art}
Zu den Meilensteinen auf diesem Gebiet der Forschung zählt sicher die Arbeit von Yu Zheng in welcher er unter anderem auf die Erkennung von den Abschnitten mit nur einem Verkehrsmittel eingeht. Weiters verwendete er in seiner Arbeit auch einen mit dem von den GPS-Spuren gesammelten geographischen Wissen aufgebauten Graphen welcher zur weiteren Auswertung verwendet wurde.\cite{zheng_understanding_2010}

Mit der Frage wie geographische Daten in eine solche Analyse miteinbezogen werden können hat sich auch Leon Stenneth beschäftigt. Dabei hat er nicht nur fixe Daten wie Gleise und Busstationen sondern auch aktuelle Buspositionen miteinbezogen. \cite{stenneth_transportation_2011}

Sowohl Stenneth als auch Zheng haben in ihren Arbeiten detailiert erklärt wieso sie welche Attribute (Geschwindigkeit, Beschleunigung, ...) für die Bestimmung des Verkehrsmittels verwendet haben und sie haben diese auch durch Versuche nach ihrer Wichtigkeit gereiht. Außerdem haben beide und auch Sasank Reddy mehrere Schlussfolgerungsmodelle (Entscheidungsbaum, Bayessches Netz, Markov Modelle, Random Forest, ...) betrachtet und miteinander verglichen. \cite{reddy_using_2010, stenneth_transportation_2011, zheng_understanding_2010}

Wie man mit Verbindungsabbrüchen umgeht und zwischen ähnlichen Verkehrsmitteln unterscheiden kann hat unter anderem auch Filip Biljecki beschäftigt. Des Weiteren baut er für die unterschiedlichen Kategorien von Verkehrsmittel ein hierarchisches Modell auf, welches ihm helfen soll bessere Entscheidungen zu treffen. \cite{biljecki_transportation_2013}

\section{Daten}
Ein wesentlicher Unterschied zwischen all den betrachteten Publikationen sind die verwendeten Daten. Einzig die GPS-Spuren bilden eine gemeinsame Basis. Manche untersuchten die Verwendung von GSM- und WIFI-Informationen \cite{reddy_using_2010},  stützten sich auf Zusatzinformationen durch weitere Sensoren wie zum Beispiel ein Beschleunigungssensor \cite{reddy_using_2010, nadine_schussler_improving_2011}. Andere wie Leon Stenneth verwendeten Live-Informationen von den öffentlichen Verkehrsmittel und kombinierten diese mit GIS-Informationen, seien es Busstationen, Bahnstrecken, das Straßennetz oder Parkplätze \cite{stenneth_transportation_2011}.

\section{Verkehrsmittel}
Ein weiterer Unterschied zwischen den Publikationen sind die betrachteten Verkehrsmittel. Hierbei reicht die Spanne der unterschiedenen Verkehrsmittel von "Gehen und Motorisiert" (siehe \cite{reddy_using_2010}) bis hin zu "Gehen, Zug, U-Bahn, Rad, Auto, Straßenbahn, Bus, Fähre, Segelboot und Flugzeug"\ (siehe \cite{biljecki_transportation_2013}).