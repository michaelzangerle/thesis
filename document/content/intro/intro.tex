\chapter{Einleitung}
Eine der derzeit wertvollsten Ressourcen die der Mensch selbst produziert, sind Daten. Diese Daten werden sowohl  gewollt aber auch ungewollt oder unbewusst mit Anderen geteilt. Das Stichwort Datenkrake ist allgegenwärtig und immer wieder in diversen Medien vertreten. Ein Beispiel dafür, dass sich mit solchen Daten aber auch gemeinnützige Ziele verfolgen lassen, soll diese Arbeit über das automatisierte Erkennen von verschiedenen Fortbewegungsmitteln sein. 

\section{Ziele dieser Arbeit}
Ziel dieser Arbeit ist es, einen Prototypen zu erstellen, welcher anhand von aufgezeichneten GPS-Spuren und in Kombination mit verschieden Methodiken versucht das benutzte Fortbewegungs- bzw. Verkehrsmittel mit einer möglichst hohen Wahrscheinlichkeit zu bestimmen. Nach dem automatischen Bestimmen der Verkehrsmittel wird eine Möglichkeit zur Korrektur der erkannten Verkehrsmittel angeboten. Diese Aufzeichnungen enthalten dabei keinerlei Informationen über die jeweilige Person. Deshalb erfolgt die Auswertung ausschließlich über die aufgezeichneten Daten sowie über öffentliche zugängliche Daten (z.B. Busstationen und Gleise). Jede dieser Aufzeichnung kann mehrere Verkehrsmittel beinhalten und von unterschiedlicher Länge sein. 

Dabei wird auch untersucht welches Level an Genauigkeit sowohl mit als auch ohne geografischen Informationen im Raum Vorarlberg erreicht werden kann. Dies soll über die Möglichkeit der manuellen Korrektur der automatischen Analyse der Verkehrsmittel geschehen. Außerdem soll eine Aussage über die Machbarkeit beziehungsweise den benötigten Aufwand und Daten gemacht werden.

In dieser Arbeit werden nicht diverse Schlussfolgerungsmodelle betrachtet und verglichen sondern es wird auf ein aufgrund von anderen Arbeiten vielversprechendes Modell gesetzt. Dieses Modell wird ein Entscheidungsbaum sein. Außerdem werden die Daten zur Analyse bzw. Auswertung nicht in Echtzeit betrachtet sondern in Form einer GPS-Spur an den Prototypen übergeben.

\section{Motivation und Nutzen}
Wird eine solche Auswertung mit einer für das Zielgebiet aussagekräftigen Anzahl an Personen durchgeführt, so kann das Resultat  für verschiedenste Zwecke verwendet werden. Auch ohne  Analyse kann rein durch die Betrachtung der gesammelten GPS-Spuren festgestellt werden, welche Routen besonders häufig benutzt werden. Zieht man nun verschiedene Werte aus der Auswertung hinzu kann auch festgestellt werden wo sich zum Beispiel verkehrstechnische Engstellen befinden oder Aussagen über die allgemeine Verkehrssituation machen. Durch die gesammelten Daten könnten sich auch Simulationen für anstehende Bauvorhaben machen lassen und auch versucht werden eine Vorhersage für bestimmte Situationen zu tätigen. Diese Aspekte können unter anderem für das Verkehrsministerium,  den öffentlichen Personennahverkehr, einen Stadtrat oder auch für die Stadtplanung (\todo{Zitat kairo} siehe Kairo) sehr interessant sein.

Eine ganze Reihe von Apps ließe sich mit den Auswertungen erstellen. Diese Apps könnten versuchen die Auswertungen in soziale Medien zu integrieren, für Fitnessanalysen verwendet werden, einen einfachen Rückblick über die eigene Fortbewegung ermöglichen oder für Umweltbewusste errechnen wie viel CO2 sie produziert oder gespart haben. Ein Reisetagebuch könnte daraus genau so Nutzen ziehen wie eine App die beim Autofahren Auskunft über die aktuell billigste Tankstelle in näherer Umgebung gibt. 

Zusammenfassen kann man sagen, dass die Verwendungsmöglichkeiten für solche Daten umfangreich sind und sich am besten unter den Begriffen kontextorientiert und geographisch zusammenfassen lassen. Nicht zuletzt öffnen sich mit solchen Daten aber auch umfangreiche Möglichkeiten für die Werbebranche.


\section{State of the art}
Zu den Meilensteinen auf diesem Gebiet zählen sicher die Arbeiten von \todo{zheng zitat}Zheng in welcher er unter anderem auf die Erkennung von den Abschnitten mit nur einem Verkehrsmittel eingeht. Weiters wurde in dieser Arbeit zur Auswertung auch ein Graph mit dem von den GPS-Spuren gesammelten geographischen Wissen aufgebaut.

Mit der Frage wie geographische Daten in eine solche Analyse miteinbezogen werden können hat sich auch  \todo{Stenneth zitat}Stenneth beschäftigt. Dabei hat er nicht nur fixe Daten wie Gleise und Busstationen sondern auch aktuelle Buspositionen betrachtet. 

Sowohl Stenneth als auch Zheng haben in ihren Arbeiten detailiert erklärt wieso sie welche Attribute (Geschwindigkeit, Beschleunigung, ...) für die Bestimmung des Verkehrsmittels verwendet haben und sie haben diese auch durch Versuche nach ihrer Wichtigkeit gereiht. Außerdem haben beide und auch Reddy \todo{reddy zitat} mehrere Schlussfolgerungsmodelle (Entscheidungsbaum, Bayessches Netz, Markov Modelle, Random Forest, ...) betrachtet und miteinander verglichen.

Wie man mit Verbindungsabbrüchen umgeht und zwischen ähnlichen Verkehrsmitteln unterscheiden kann hat unter anderem auch \todo{Biljeki zitat}Biljecki beschäftigt. Außerdem baut er für die unterschiedlichen Kategorien von Verkehrsmittel ein hierarchisches Modell auf welches ihm helfen soll bessere Entscheidungen zu treffen.

\section{Weiterer Aufbau der Arbeit}
